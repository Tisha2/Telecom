\documentclass[11pt]{article}

    \usepackage[breakable]{tcolorbox}
    \usepackage{parskip} % Stop auto-indenting (to mimic markdown behaviour)
    
    \usepackage{iftex}
    \ifPDFTeX
    	\usepackage[T1]{fontenc}
    	\usepackage{mathpazo}
    \else
    	\usepackage{fontspec}
    \fi

    % Basic figure setup, for now with no caption control since it's done
    % automatically by Pandoc (which extracts ![](path) syntax from Markdown).
    \usepackage{graphicx}
    % Maintain compatibility with old templates. Remove in nbconvert 6.0
    \let\Oldincludegraphics\includegraphics
    % Ensure that by default, figures have no caption (until we provide a
    % proper Figure object with a Caption API and a way to capture that
    % in the conversion process - todo).
    \usepackage{caption}
    \DeclareCaptionFormat{nocaption}{}
    \captionsetup{format=nocaption,aboveskip=0pt,belowskip=0pt}

    \usepackage{float}
    \floatplacement{figure}{H} % forces figures to be placed at the correct location
    \usepackage{xcolor} % Allow colors to be defined
    \usepackage{enumerate} % Needed for markdown enumerations to work
    \usepackage{geometry} % Used to adjust the document margins
    \usepackage{amsmath} % Equations
    \usepackage{amssymb} % Equations
    \usepackage{textcomp} % defines textquotesingle
    % Hack from http://tex.stackexchange.com/a/47451/13684:
    \AtBeginDocument{%
        \def\PYZsq{\textquotesingle}% Upright quotes in Pygmentized code
    }
    \usepackage{upquote} % Upright quotes for verbatim code
    \usepackage{eurosym} % defines \euro
    \usepackage[mathletters]{ucs} % Extended unicode (utf-8) support
    \usepackage{fancyvrb} % verbatim replacement that allows latex
    \usepackage{grffile} % extends the file name processing of package graphics 
                         % to support a larger range
    \makeatletter % fix for old versions of grffile with XeLaTeX
    \@ifpackagelater{grffile}{2019/11/01}
    {
      % Do nothing on new versions
    }
    {
      \def\Gread@@xetex#1{%
        \IfFileExists{"\Gin@base".bb}%
        {\Gread@eps{\Gin@base.bb}}%
        {\Gread@@xetex@aux#1}%
      }
    }
    \makeatother
    \usepackage[Export]{adjustbox} % Used to constrain images to a maximum size
    \adjustboxset{max size={0.9\linewidth}{0.9\paperheight}}

    % The hyperref package gives us a pdf with properly built
    % internal navigation ('pdf bookmarks' for the table of contents,
    % internal cross-reference links, web links for URLs, etc.)
    \usepackage{hyperref}
    % The default LaTeX title has an obnoxious amount of whitespace. By default,
    % titling removes some of it. It also provides customization options.
    \usepackage{titling}
    \usepackage{longtable} % longtable support required by pandoc >1.10
    \usepackage{booktabs}  % table support for pandoc > 1.12.2
    \usepackage[inline]{enumitem} % IRkernel/repr support (it uses the enumerate* environment)
    \usepackage[normalem]{ulem} % ulem is needed to support strikethroughs (\sout)
                                % normalem makes italics be italics, not underlines
    \usepackage{mathrsfs}
    

    
    % Colors for the hyperref package
    \definecolor{urlcolor}{rgb}{0,.145,.698}
    \definecolor{linkcolor}{rgb}{.71,0.21,0.01}
    \definecolor{citecolor}{rgb}{.12,.54,.11}

    % ANSI colors
    \definecolor{ansi-black}{HTML}{3E424D}
    \definecolor{ansi-black-intense}{HTML}{282C36}
    \definecolor{ansi-red}{HTML}{E75C58}
    \definecolor{ansi-red-intense}{HTML}{B22B31}
    \definecolor{ansi-green}{HTML}{00A250}
    \definecolor{ansi-green-intense}{HTML}{007427}
    \definecolor{ansi-yellow}{HTML}{DDB62B}
    \definecolor{ansi-yellow-intense}{HTML}{B27D12}
    \definecolor{ansi-blue}{HTML}{208FFB}
    \definecolor{ansi-blue-intense}{HTML}{0065CA}
    \definecolor{ansi-magenta}{HTML}{D160C4}
    \definecolor{ansi-magenta-intense}{HTML}{A03196}
    \definecolor{ansi-cyan}{HTML}{60C6C8}
    \definecolor{ansi-cyan-intense}{HTML}{258F8F}
    \definecolor{ansi-white}{HTML}{C5C1B4}
    \definecolor{ansi-white-intense}{HTML}{A1A6B2}
    \definecolor{ansi-default-inverse-fg}{HTML}{FFFFFF}
    \definecolor{ansi-default-inverse-bg}{HTML}{000000}

    % common color for the border for error outputs.
    \definecolor{outerrorbackground}{HTML}{FFDFDF}

    % commands and environments needed by pandoc snippets
    % extracted from the output of `pandoc -s`
    \providecommand{\tightlist}{%
      \setlength{\itemsep}{0pt}\setlength{\parskip}{0pt}}
    \DefineVerbatimEnvironment{Highlighting}{Verbatim}{commandchars=\\\{\}}
    % Add ',fontsize=\small' for more characters per line
    \newenvironment{Shaded}{}{}
    \newcommand{\KeywordTok}[1]{\textcolor[rgb]{0.00,0.44,0.13}{\textbf{{#1}}}}
    \newcommand{\DataTypeTok}[1]{\textcolor[rgb]{0.56,0.13,0.00}{{#1}}}
    \newcommand{\DecValTok}[1]{\textcolor[rgb]{0.25,0.63,0.44}{{#1}}}
    \newcommand{\BaseNTok}[1]{\textcolor[rgb]{0.25,0.63,0.44}{{#1}}}
    \newcommand{\FloatTok}[1]{\textcolor[rgb]{0.25,0.63,0.44}{{#1}}}
    \newcommand{\CharTok}[1]{\textcolor[rgb]{0.25,0.44,0.63}{{#1}}}
    \newcommand{\StringTok}[1]{\textcolor[rgb]{0.25,0.44,0.63}{{#1}}}
    \newcommand{\CommentTok}[1]{\textcolor[rgb]{0.38,0.63,0.69}{\textit{{#1}}}}
    \newcommand{\OtherTok}[1]{\textcolor[rgb]{0.00,0.44,0.13}{{#1}}}
    \newcommand{\AlertTok}[1]{\textcolor[rgb]{1.00,0.00,0.00}{\textbf{{#1}}}}
    \newcommand{\FunctionTok}[1]{\textcolor[rgb]{0.02,0.16,0.49}{{#1}}}
    \newcommand{\RegionMarkerTok}[1]{{#1}}
    \newcommand{\ErrorTok}[1]{\textcolor[rgb]{1.00,0.00,0.00}{\textbf{{#1}}}}
    \newcommand{\NormalTok}[1]{{#1}}
    
    % Additional commands for more recent versions of Pandoc
    \newcommand{\ConstantTok}[1]{\textcolor[rgb]{0.53,0.00,0.00}{{#1}}}
    \newcommand{\SpecialCharTok}[1]{\textcolor[rgb]{0.25,0.44,0.63}{{#1}}}
    \newcommand{\VerbatimStringTok}[1]{\textcolor[rgb]{0.25,0.44,0.63}{{#1}}}
    \newcommand{\SpecialStringTok}[1]{\textcolor[rgb]{0.73,0.40,0.53}{{#1}}}
    \newcommand{\ImportTok}[1]{{#1}}
    \newcommand{\DocumentationTok}[1]{\textcolor[rgb]{0.73,0.13,0.13}{\textit{{#1}}}}
    \newcommand{\AnnotationTok}[1]{\textcolor[rgb]{0.38,0.63,0.69}{\textbf{\textit{{#1}}}}}
    \newcommand{\CommentVarTok}[1]{\textcolor[rgb]{0.38,0.63,0.69}{\textbf{\textit{{#1}}}}}
    \newcommand{\VariableTok}[1]{\textcolor[rgb]{0.10,0.09,0.49}{{#1}}}
    \newcommand{\ControlFlowTok}[1]{\textcolor[rgb]{0.00,0.44,0.13}{\textbf{{#1}}}}
    \newcommand{\OperatorTok}[1]{\textcolor[rgb]{0.40,0.40,0.40}{{#1}}}
    \newcommand{\BuiltInTok}[1]{{#1}}
    \newcommand{\ExtensionTok}[1]{{#1}}
    \newcommand{\PreprocessorTok}[1]{\textcolor[rgb]{0.74,0.48,0.00}{{#1}}}
    \newcommand{\AttributeTok}[1]{\textcolor[rgb]{0.49,0.56,0.16}{{#1}}}
    \newcommand{\InformationTok}[1]{\textcolor[rgb]{0.38,0.63,0.69}{\textbf{\textit{{#1}}}}}
    \newcommand{\WarningTok}[1]{\textcolor[rgb]{0.38,0.63,0.69}{\textbf{\textit{{#1}}}}}
    
    
    % Define a nice break command that doesn't care if a line doesn't already
    % exist.
    \def\br{\hspace*{\fill} \\* }
    % Math Jax compatibility definitions
    \def\gt{>}
    \def\lt{<}
    \let\Oldtex\TeX
    \let\Oldlatex\LaTeX
    \renewcommand{\TeX}{\textrm{\Oldtex}}
    \renewcommand{\LaTeX}{\textrm{\Oldlatex}}
    % Document parameters
    % Document title
    \title{chap04s}
    
    
    
    
    
% Pygments definitions
\makeatletter
\def\PY@reset{\let\PY@it=\relax \let\PY@bf=\relax%
    \let\PY@ul=\relax \let\PY@tc=\relax%
    \let\PY@bc=\relax \let\PY@ff=\relax}
\def\PY@tok#1{\csname PY@tok@#1\endcsname}
\def\PY@toks#1+{\ifx\relax#1\empty\else%
    \PY@tok{#1}\expandafter\PY@toks\fi}
\def\PY@do#1{\PY@bc{\PY@tc{\PY@ul{%
    \PY@it{\PY@bf{\PY@ff{#1}}}}}}}
\def\PY#1#2{\PY@reset\PY@toks#1+\relax+\PY@do{#2}}

\@namedef{PY@tok@w}{\def\PY@tc##1{\textcolor[rgb]{0.73,0.73,0.73}{##1}}}
\@namedef{PY@tok@c}{\let\PY@it=\textit\def\PY@tc##1{\textcolor[rgb]{0.25,0.50,0.50}{##1}}}
\@namedef{PY@tok@cp}{\def\PY@tc##1{\textcolor[rgb]{0.74,0.48,0.00}{##1}}}
\@namedef{PY@tok@k}{\let\PY@bf=\textbf\def\PY@tc##1{\textcolor[rgb]{0.00,0.50,0.00}{##1}}}
\@namedef{PY@tok@kp}{\def\PY@tc##1{\textcolor[rgb]{0.00,0.50,0.00}{##1}}}
\@namedef{PY@tok@kt}{\def\PY@tc##1{\textcolor[rgb]{0.69,0.00,0.25}{##1}}}
\@namedef{PY@tok@o}{\def\PY@tc##1{\textcolor[rgb]{0.40,0.40,0.40}{##1}}}
\@namedef{PY@tok@ow}{\let\PY@bf=\textbf\def\PY@tc##1{\textcolor[rgb]{0.67,0.13,1.00}{##1}}}
\@namedef{PY@tok@nb}{\def\PY@tc##1{\textcolor[rgb]{0.00,0.50,0.00}{##1}}}
\@namedef{PY@tok@nf}{\def\PY@tc##1{\textcolor[rgb]{0.00,0.00,1.00}{##1}}}
\@namedef{PY@tok@nc}{\let\PY@bf=\textbf\def\PY@tc##1{\textcolor[rgb]{0.00,0.00,1.00}{##1}}}
\@namedef{PY@tok@nn}{\let\PY@bf=\textbf\def\PY@tc##1{\textcolor[rgb]{0.00,0.00,1.00}{##1}}}
\@namedef{PY@tok@ne}{\let\PY@bf=\textbf\def\PY@tc##1{\textcolor[rgb]{0.82,0.25,0.23}{##1}}}
\@namedef{PY@tok@nv}{\def\PY@tc##1{\textcolor[rgb]{0.10,0.09,0.49}{##1}}}
\@namedef{PY@tok@no}{\def\PY@tc##1{\textcolor[rgb]{0.53,0.00,0.00}{##1}}}
\@namedef{PY@tok@nl}{\def\PY@tc##1{\textcolor[rgb]{0.63,0.63,0.00}{##1}}}
\@namedef{PY@tok@ni}{\let\PY@bf=\textbf\def\PY@tc##1{\textcolor[rgb]{0.60,0.60,0.60}{##1}}}
\@namedef{PY@tok@na}{\def\PY@tc##1{\textcolor[rgb]{0.49,0.56,0.16}{##1}}}
\@namedef{PY@tok@nt}{\let\PY@bf=\textbf\def\PY@tc##1{\textcolor[rgb]{0.00,0.50,0.00}{##1}}}
\@namedef{PY@tok@nd}{\def\PY@tc##1{\textcolor[rgb]{0.67,0.13,1.00}{##1}}}
\@namedef{PY@tok@s}{\def\PY@tc##1{\textcolor[rgb]{0.73,0.13,0.13}{##1}}}
\@namedef{PY@tok@sd}{\let\PY@it=\textit\def\PY@tc##1{\textcolor[rgb]{0.73,0.13,0.13}{##1}}}
\@namedef{PY@tok@si}{\let\PY@bf=\textbf\def\PY@tc##1{\textcolor[rgb]{0.73,0.40,0.53}{##1}}}
\@namedef{PY@tok@se}{\let\PY@bf=\textbf\def\PY@tc##1{\textcolor[rgb]{0.73,0.40,0.13}{##1}}}
\@namedef{PY@tok@sr}{\def\PY@tc##1{\textcolor[rgb]{0.73,0.40,0.53}{##1}}}
\@namedef{PY@tok@ss}{\def\PY@tc##1{\textcolor[rgb]{0.10,0.09,0.49}{##1}}}
\@namedef{PY@tok@sx}{\def\PY@tc##1{\textcolor[rgb]{0.00,0.50,0.00}{##1}}}
\@namedef{PY@tok@m}{\def\PY@tc##1{\textcolor[rgb]{0.40,0.40,0.40}{##1}}}
\@namedef{PY@tok@gh}{\let\PY@bf=\textbf\def\PY@tc##1{\textcolor[rgb]{0.00,0.00,0.50}{##1}}}
\@namedef{PY@tok@gu}{\let\PY@bf=\textbf\def\PY@tc##1{\textcolor[rgb]{0.50,0.00,0.50}{##1}}}
\@namedef{PY@tok@gd}{\def\PY@tc##1{\textcolor[rgb]{0.63,0.00,0.00}{##1}}}
\@namedef{PY@tok@gi}{\def\PY@tc##1{\textcolor[rgb]{0.00,0.63,0.00}{##1}}}
\@namedef{PY@tok@gr}{\def\PY@tc##1{\textcolor[rgb]{1.00,0.00,0.00}{##1}}}
\@namedef{PY@tok@ge}{\let\PY@it=\textit}
\@namedef{PY@tok@gs}{\let\PY@bf=\textbf}
\@namedef{PY@tok@gp}{\let\PY@bf=\textbf\def\PY@tc##1{\textcolor[rgb]{0.00,0.00,0.50}{##1}}}
\@namedef{PY@tok@go}{\def\PY@tc##1{\textcolor[rgb]{0.53,0.53,0.53}{##1}}}
\@namedef{PY@tok@gt}{\def\PY@tc##1{\textcolor[rgb]{0.00,0.27,0.87}{##1}}}
\@namedef{PY@tok@err}{\def\PY@bc##1{{\setlength{\fboxsep}{\string -\fboxrule}\fcolorbox[rgb]{1.00,0.00,0.00}{1,1,1}{\strut ##1}}}}
\@namedef{PY@tok@kc}{\let\PY@bf=\textbf\def\PY@tc##1{\textcolor[rgb]{0.00,0.50,0.00}{##1}}}
\@namedef{PY@tok@kd}{\let\PY@bf=\textbf\def\PY@tc##1{\textcolor[rgb]{0.00,0.50,0.00}{##1}}}
\@namedef{PY@tok@kn}{\let\PY@bf=\textbf\def\PY@tc##1{\textcolor[rgb]{0.00,0.50,0.00}{##1}}}
\@namedef{PY@tok@kr}{\let\PY@bf=\textbf\def\PY@tc##1{\textcolor[rgb]{0.00,0.50,0.00}{##1}}}
\@namedef{PY@tok@bp}{\def\PY@tc##1{\textcolor[rgb]{0.00,0.50,0.00}{##1}}}
\@namedef{PY@tok@fm}{\def\PY@tc##1{\textcolor[rgb]{0.00,0.00,1.00}{##1}}}
\@namedef{PY@tok@vc}{\def\PY@tc##1{\textcolor[rgb]{0.10,0.09,0.49}{##1}}}
\@namedef{PY@tok@vg}{\def\PY@tc##1{\textcolor[rgb]{0.10,0.09,0.49}{##1}}}
\@namedef{PY@tok@vi}{\def\PY@tc##1{\textcolor[rgb]{0.10,0.09,0.49}{##1}}}
\@namedef{PY@tok@vm}{\def\PY@tc##1{\textcolor[rgb]{0.10,0.09,0.49}{##1}}}
\@namedef{PY@tok@sa}{\def\PY@tc##1{\textcolor[rgb]{0.73,0.13,0.13}{##1}}}
\@namedef{PY@tok@sb}{\def\PY@tc##1{\textcolor[rgb]{0.73,0.13,0.13}{##1}}}
\@namedef{PY@tok@sc}{\def\PY@tc##1{\textcolor[rgb]{0.73,0.13,0.13}{##1}}}
\@namedef{PY@tok@dl}{\def\PY@tc##1{\textcolor[rgb]{0.73,0.13,0.13}{##1}}}
\@namedef{PY@tok@s2}{\def\PY@tc##1{\textcolor[rgb]{0.73,0.13,0.13}{##1}}}
\@namedef{PY@tok@sh}{\def\PY@tc##1{\textcolor[rgb]{0.73,0.13,0.13}{##1}}}
\@namedef{PY@tok@s1}{\def\PY@tc##1{\textcolor[rgb]{0.73,0.13,0.13}{##1}}}
\@namedef{PY@tok@mb}{\def\PY@tc##1{\textcolor[rgb]{0.40,0.40,0.40}{##1}}}
\@namedef{PY@tok@mf}{\def\PY@tc##1{\textcolor[rgb]{0.40,0.40,0.40}{##1}}}
\@namedef{PY@tok@mh}{\def\PY@tc##1{\textcolor[rgb]{0.40,0.40,0.40}{##1}}}
\@namedef{PY@tok@mi}{\def\PY@tc##1{\textcolor[rgb]{0.40,0.40,0.40}{##1}}}
\@namedef{PY@tok@il}{\def\PY@tc##1{\textcolor[rgb]{0.40,0.40,0.40}{##1}}}
\@namedef{PY@tok@mo}{\def\PY@tc##1{\textcolor[rgb]{0.40,0.40,0.40}{##1}}}
\@namedef{PY@tok@ch}{\let\PY@it=\textit\def\PY@tc##1{\textcolor[rgb]{0.25,0.50,0.50}{##1}}}
\@namedef{PY@tok@cm}{\let\PY@it=\textit\def\PY@tc##1{\textcolor[rgb]{0.25,0.50,0.50}{##1}}}
\@namedef{PY@tok@cpf}{\let\PY@it=\textit\def\PY@tc##1{\textcolor[rgb]{0.25,0.50,0.50}{##1}}}
\@namedef{PY@tok@c1}{\let\PY@it=\textit\def\PY@tc##1{\textcolor[rgb]{0.25,0.50,0.50}{##1}}}
\@namedef{PY@tok@cs}{\let\PY@it=\textit\def\PY@tc##1{\textcolor[rgb]{0.25,0.50,0.50}{##1}}}

\def\PYZbs{\char`\\}
\def\PYZus{\char`\_}
\def\PYZob{\char`\{}
\def\PYZcb{\char`\}}
\def\PYZca{\char`\^}
\def\PYZam{\char`\&}
\def\PYZlt{\char`\<}
\def\PYZgt{\char`\>}
\def\PYZsh{\char`\#}
\def\PYZpc{\char`\%}
\def\PYZdl{\char`\$}
\def\PYZhy{\char`\-}
\def\PYZsq{\char`\'}
\def\PYZdq{\char`\"}
\def\PYZti{\char`\~}
% for compatibility with earlier versions
\def\PYZat{@}
\def\PYZlb{[}
\def\PYZrb{]}
\makeatother


    % For linebreaks inside Verbatim environment from package fancyvrb. 
    \makeatletter
        \newbox\Wrappedcontinuationbox 
        \newbox\Wrappedvisiblespacebox 
        \newcommand*\Wrappedvisiblespace {\textcolor{red}{\textvisiblespace}} 
        \newcommand*\Wrappedcontinuationsymbol {\textcolor{red}{\llap{\tiny$\m@th\hookrightarrow$}}} 
        \newcommand*\Wrappedcontinuationindent {3ex } 
        \newcommand*\Wrappedafterbreak {\kern\Wrappedcontinuationindent\copy\Wrappedcontinuationbox} 
        % Take advantage of the already applied Pygments mark-up to insert 
        % potential linebreaks for TeX processing. 
        %        {, <, #, %, $, ' and ": go to next line. 
        %        _, }, ^, &, >, - and ~: stay at end of broken line. 
        % Use of \textquotesingle for straight quote. 
        \newcommand*\Wrappedbreaksatspecials {% 
            \def\PYGZus{\discretionary{\char`\_}{\Wrappedafterbreak}{\char`\_}}% 
            \def\PYGZob{\discretionary{}{\Wrappedafterbreak\char`\{}{\char`\{}}% 
            \def\PYGZcb{\discretionary{\char`\}}{\Wrappedafterbreak}{\char`\}}}% 
            \def\PYGZca{\discretionary{\char`\^}{\Wrappedafterbreak}{\char`\^}}% 
            \def\PYGZam{\discretionary{\char`\&}{\Wrappedafterbreak}{\char`\&}}% 
            \def\PYGZlt{\discretionary{}{\Wrappedafterbreak\char`\<}{\char`\<}}% 
            \def\PYGZgt{\discretionary{\char`\>}{\Wrappedafterbreak}{\char`\>}}% 
            \def\PYGZsh{\discretionary{}{\Wrappedafterbreak\char`\#}{\char`\#}}% 
            \def\PYGZpc{\discretionary{}{\Wrappedafterbreak\char`\%}{\char`\%}}% 
            \def\PYGZdl{\discretionary{}{\Wrappedafterbreak\char`\$}{\char`\$}}% 
            \def\PYGZhy{\discretionary{\char`\-}{\Wrappedafterbreak}{\char`\-}}% 
            \def\PYGZsq{\discretionary{}{\Wrappedafterbreak\textquotesingle}{\textquotesingle}}% 
            \def\PYGZdq{\discretionary{}{\Wrappedafterbreak\char`\"}{\char`\"}}% 
            \def\PYGZti{\discretionary{\char`\~}{\Wrappedafterbreak}{\char`\~}}% 
        } 
        % Some characters . , ; ? ! / are not pygmentized. 
        % This macro makes them "active" and they will insert potential linebreaks 
        \newcommand*\Wrappedbreaksatpunct {% 
            \lccode`\~`\.\lowercase{\def~}{\discretionary{\hbox{\char`\.}}{\Wrappedafterbreak}{\hbox{\char`\.}}}% 
            \lccode`\~`\,\lowercase{\def~}{\discretionary{\hbox{\char`\,}}{\Wrappedafterbreak}{\hbox{\char`\,}}}% 
            \lccode`\~`\;\lowercase{\def~}{\discretionary{\hbox{\char`\;}}{\Wrappedafterbreak}{\hbox{\char`\;}}}% 
            \lccode`\~`\:\lowercase{\def~}{\discretionary{\hbox{\char`\:}}{\Wrappedafterbreak}{\hbox{\char`\:}}}% 
            \lccode`\~`\?\lowercase{\def~}{\discretionary{\hbox{\char`\?}}{\Wrappedafterbreak}{\hbox{\char`\?}}}% 
            \lccode`\~`\!\lowercase{\def~}{\discretionary{\hbox{\char`\!}}{\Wrappedafterbreak}{\hbox{\char`\!}}}% 
            \lccode`\~`\/\lowercase{\def~}{\discretionary{\hbox{\char`\/}}{\Wrappedafterbreak}{\hbox{\char`\/}}}% 
            \catcode`\.\active
            \catcode`\,\active 
            \catcode`\;\active
            \catcode`\:\active
            \catcode`\?\active
            \catcode`\!\active
            \catcode`\/\active 
            \lccode`\~`\~ 	
        }
    \makeatother

    \let\OriginalVerbatim=\Verbatim
    \makeatletter
    \renewcommand{\Verbatim}[1][1]{%
        %\parskip\z@skip
        \sbox\Wrappedcontinuationbox {\Wrappedcontinuationsymbol}%
        \sbox\Wrappedvisiblespacebox {\FV@SetupFont\Wrappedvisiblespace}%
        \def\FancyVerbFormatLine ##1{\hsize\linewidth
            \vtop{\raggedright\hyphenpenalty\z@\exhyphenpenalty\z@
                \doublehyphendemerits\z@\finalhyphendemerits\z@
                \strut ##1\strut}%
        }%
        % If the linebreak is at a space, the latter will be displayed as visible
        % space at end of first line, and a continuation symbol starts next line.
        % Stretch/shrink are however usually zero for typewriter font.
        \def\FV@Space {%
            \nobreak\hskip\z@ plus\fontdimen3\font minus\fontdimen4\font
            \discretionary{\copy\Wrappedvisiblespacebox}{\Wrappedafterbreak}
            {\kern\fontdimen2\font}%
        }%
        
        % Allow breaks at special characters using \PYG... macros.
        \Wrappedbreaksatspecials
        % Breaks at punctuation characters . , ; ? ! and / need catcode=\active 	
        \OriginalVerbatim[#1,codes*=\Wrappedbreaksatpunct]%
    }
    \makeatother

    % Exact colors from NB
    \definecolor{incolor}{HTML}{303F9F}
    \definecolor{outcolor}{HTML}{D84315}
    \definecolor{cellborder}{HTML}{CFCFCF}
    \definecolor{cellbackground}{HTML}{F7F7F7}
    
    % prompt
    \makeatletter
    \newcommand{\boxspacing}{\kern\kvtcb@left@rule\kern\kvtcb@boxsep}
    \makeatother
    \newcommand{\prompt}[4]{
        {\ttfamily\llap{{\color{#2}[#3]:\hspace{3pt}#4}}\vspace{-\baselineskip}}
    }
    

    
    % Prevent overflowing lines due to hard-to-break entities
    \sloppy 
    % Setup hyperref package
    \hypersetup{
      breaklinks=true,  % so long urls are correctly broken across lines
      colorlinks=true,
      urlcolor=urlcolor,
      linkcolor=linkcolor,
      citecolor=citecolor,
      }
    % Slightly bigger margins than the latex defaults
    
    \geometry{verbose,tmargin=1in,bmargin=1in,lmargin=1in,rmargin=1in}
    
    

\begin{document}
    
    \maketitle
    
    

    
    \hypertarget{thinkdsp}{%
\subsection{ThinkDSP}\label{thinkdsp}}

This notebook contains solutions to exercises in Chapter 4: Noise

Copyright 2015 Allen Downey

License: \href{http://creativecommons.org/licenses/by/4.0/}{Creative
Commons Attribution 4.0 International}

    \begin{tcolorbox}[breakable, size=fbox, boxrule=1pt, pad at break*=1mm,colback=cellbackground, colframe=cellborder]
\prompt{In}{incolor}{1}{\boxspacing}
\begin{Verbatim}[commandchars=\\\{\}]
\PY{c+c1}{\PYZsh{} Get thinkdsp.py}
\end{Verbatim}
\end{tcolorbox}

    \begin{tcolorbox}[breakable, size=fbox, boxrule=1pt, pad at break*=1mm,colback=cellbackground, colframe=cellborder]
\prompt{In}{incolor}{2}{\boxspacing}
\begin{Verbatim}[commandchars=\\\{\}]
\PY{k+kn}{import} \PY{n+nn}{numpy} \PY{k}{as} \PY{n+nn}{np}
\PY{k+kn}{import} \PY{n+nn}{matplotlib}\PY{n+nn}{.}\PY{n+nn}{pyplot} \PY{k}{as} \PY{n+nn}{plt}

\PY{k+kn}{from} \PY{n+nn}{thinkdsp} \PY{k+kn}{import} \PY{n}{decorate}
\end{Verbatim}
\end{tcolorbox}

    \hypertarget{exercise-1}{%
\subsection{Exercise 1}\label{exercise-1}}

``A Soft Murmur'\,' is a web site that plays a mixture of natural noise
sources, including rain, waves, wind, etc. At
http://asoftmurmur.com/about/ you can find their list of recordings,
most of which are at http://freesound.org.

Download a few of these files and compute the spectrum of each signal.
Does the power spectrum look like white noise, pink noise, or Brownian
noise? How does the spectrum vary over time?

    \begin{tcolorbox}[breakable, size=fbox, boxrule=1pt, pad at break*=1mm,colback=cellbackground, colframe=cellborder]
\prompt{In}{incolor}{3}{\boxspacing}
\begin{Verbatim}[commandchars=\\\{\}]
\PY{k+kn}{from} \PY{n+nn}{thinkdsp} \PY{k+kn}{import} \PY{n}{read\PYZus{}wave}

\PY{n}{wave} \PY{o}{=} \PY{n}{read\PYZus{}wave}\PY{p}{(}\PY{l+s+s1}{\PYZsq{}}\PY{l+s+s1}{Sounds/401275\PYZus{}\PYZus{}inspectorj\PYZus{}\PYZus{}rain\PYZhy{}moderate\PYZhy{}c.wav}\PY{l+s+s1}{\PYZsq{}}\PY{p}{)}
\PY{n}{wave}\PY{o}{.}\PY{n}{make\PYZus{}audio}\PY{p}{(}\PY{p}{)}
\end{Verbatim}
\end{tcolorbox}

            \begin{tcolorbox}[breakable, size=fbox, boxrule=.5pt, pad at break*=1mm, opacityfill=0]
\prompt{Out}{outcolor}{3}{\boxspacing}
\begin{Verbatim}[commandchars=\\\{\}]
<IPython.lib.display.Audio object>
\end{Verbatim}
\end{tcolorbox}
        
    \begin{tcolorbox}[breakable, size=fbox, boxrule=1pt, pad at break*=1mm,colback=cellbackground, colframe=cellborder]
\prompt{In}{incolor}{4}{\boxspacing}
\begin{Verbatim}[commandchars=\\\{\}]
\PY{n}{segment} \PY{o}{=} \PY{n}{wave}\PY{o}{.}\PY{n}{segment}\PY{p}{(}\PY{n}{start}\PY{o}{=}\PY{l+m+mf}{1.5}\PY{p}{,} \PY{n}{duration}\PY{o}{=}\PY{l+m+mf}{0.005}\PY{p}{)}
\PY{n}{segment}\PY{o}{.}\PY{n}{plot}\PY{p}{(}\PY{p}{)}
\PY{n}{decorate}\PY{p}{(}\PY{n}{xlabel}\PY{o}{=}\PY{l+s+s1}{\PYZsq{}}\PY{l+s+s1}{Time (s)}\PY{l+s+s1}{\PYZsq{}}\PY{p}{,}
         \PY{n}{ylabel}\PY{o}{=}\PY{l+s+s1}{\PYZsq{}}\PY{l+s+s1}{Amplitude}\PY{l+s+s1}{\PYZsq{}}\PY{p}{)}
\end{Verbatim}
\end{tcolorbox}

    \begin{center}
    \adjustimage{max size={0.9\linewidth}{0.9\paperheight}}{output_5_0.png}
    \end{center}
    { \hspace*{\fill} \\}
    
    I chose a recording of rain waves. I selected a short segment:

    \begin{tcolorbox}[breakable, size=fbox, boxrule=1pt, pad at break*=1mm,colback=cellbackground, colframe=cellborder]
\prompt{In}{incolor}{5}{\boxspacing}
\begin{Verbatim}[commandchars=\\\{\}]
\PY{n}{segment} \PY{o}{=} \PY{n}{wave}\PY{o}{.}\PY{n}{segment}\PY{p}{(}\PY{n}{start}\PY{o}{=}\PY{l+m+mf}{1.5}\PY{p}{,} \PY{n}{duration}\PY{o}{=}\PY{l+m+mf}{1.0}\PY{p}{)}
\PY{n}{segment}\PY{o}{.}\PY{n}{make\PYZus{}audio}\PY{p}{(}\PY{p}{)}
\end{Verbatim}
\end{tcolorbox}

            \begin{tcolorbox}[breakable, size=fbox, boxrule=.5pt, pad at break*=1mm, opacityfill=0]
\prompt{Out}{outcolor}{5}{\boxspacing}
\begin{Verbatim}[commandchars=\\\{\}]
<IPython.lib.display.Audio object>
\end{Verbatim}
\end{tcolorbox}
        
    And here's its spectrum:

    \begin{tcolorbox}[breakable, size=fbox, boxrule=1pt, pad at break*=1mm,colback=cellbackground, colframe=cellborder]
\prompt{In}{incolor}{6}{\boxspacing}
\begin{Verbatim}[commandchars=\\\{\}]
\PY{n}{spectrum} \PY{o}{=} \PY{n}{segment}\PY{o}{.}\PY{n}{make\PYZus{}spectrum}\PY{p}{(}\PY{p}{)}
\PY{n}{spectrum}\PY{o}{.}\PY{n}{plot\PYZus{}power}\PY{p}{(}\PY{p}{)}
\PY{n}{decorate}\PY{p}{(}\PY{n}{xlabel}\PY{o}{=}\PY{l+s+s1}{\PYZsq{}}\PY{l+s+s1}{Frequency (Hz)}\PY{l+s+s1}{\PYZsq{}}\PY{p}{,}
         \PY{n}{ylabel}\PY{o}{=}\PY{l+s+s1}{\PYZsq{}}\PY{l+s+s1}{Power}\PY{l+s+s1}{\PYZsq{}}\PY{p}{)}
\end{Verbatim}
\end{tcolorbox}

    \begin{center}
    \adjustimage{max size={0.9\linewidth}{0.9\paperheight}}{output_9_0.png}
    \end{center}
    { \hspace*{\fill} \\}
    
    Amplitude drops off with frequency, so this might be red or pink noise.
We can check by looking at the power spectrum on a log-log scale.

    \begin{tcolorbox}[breakable, size=fbox, boxrule=1pt, pad at break*=1mm,colback=cellbackground, colframe=cellborder]
\prompt{In}{incolor}{7}{\boxspacing}
\begin{Verbatim}[commandchars=\\\{\}]
\PY{n}{spectrum}\PY{o}{.}\PY{n}{plot\PYZus{}power}\PY{p}{(}\PY{p}{)}

\PY{n}{loglog} \PY{o}{=} \PY{n+nb}{dict}\PY{p}{(}\PY{n}{xscale}\PY{o}{=}\PY{l+s+s1}{\PYZsq{}}\PY{l+s+s1}{log}\PY{l+s+s1}{\PYZsq{}}\PY{p}{,} \PY{n}{yscale}\PY{o}{=}\PY{l+s+s1}{\PYZsq{}}\PY{l+s+s1}{log}\PY{l+s+s1}{\PYZsq{}}\PY{p}{)}
\PY{n}{decorate}\PY{p}{(}\PY{n}{xlabel}\PY{o}{=}\PY{l+s+s1}{\PYZsq{}}\PY{l+s+s1}{Frequency (Hz)}\PY{l+s+s1}{\PYZsq{}}\PY{p}{,}
         \PY{n}{ylabel}\PY{o}{=}\PY{l+s+s1}{\PYZsq{}}\PY{l+s+s1}{Power}\PY{l+s+s1}{\PYZsq{}}\PY{p}{,} 
         \PY{o}{*}\PY{o}{*}\PY{n}{loglog}\PY{p}{)}
\end{Verbatim}
\end{tcolorbox}

    \begin{center}
    \adjustimage{max size={0.9\linewidth}{0.9\paperheight}}{output_11_0.png}
    \end{center}
    { \hspace*{\fill} \\}
    
    This structure, with increasing and then decreasing amplitude, seems to
be common in natural noise sources.

Above \(f = 10^3\), it might be dropping off linearly, but we can't
really tell.

To see how the spectrum changes over time, I'll select another segment:

    \begin{tcolorbox}[breakable, size=fbox, boxrule=1pt, pad at break*=1mm,colback=cellbackground, colframe=cellborder]
\prompt{In}{incolor}{8}{\boxspacing}
\begin{Verbatim}[commandchars=\\\{\}]
\PY{n}{segment2} \PY{o}{=} \PY{n}{wave}\PY{o}{.}\PY{n}{segment}\PY{p}{(}\PY{n}{start}\PY{o}{=}\PY{l+m+mi}{5}\PY{p}{,} \PY{n}{duration}\PY{o}{=}\PY{l+m+mf}{1.0}\PY{p}{)}
\PY{n}{segment2}\PY{o}{.}\PY{n}{make\PYZus{}audio}\PY{p}{(}\PY{p}{)}
\end{Verbatim}
\end{tcolorbox}

            \begin{tcolorbox}[breakable, size=fbox, boxrule=.5pt, pad at break*=1mm, opacityfill=0]
\prompt{Out}{outcolor}{8}{\boxspacing}
\begin{Verbatim}[commandchars=\\\{\}]
<IPython.lib.display.Audio object>
\end{Verbatim}
\end{tcolorbox}
        
    And plot the two spectrums:

    \begin{tcolorbox}[breakable, size=fbox, boxrule=1pt, pad at break*=1mm,colback=cellbackground, colframe=cellborder]
\prompt{In}{incolor}{9}{\boxspacing}
\begin{Verbatim}[commandchars=\\\{\}]
\PY{n}{spectrum2} \PY{o}{=} \PY{n}{segment2}\PY{o}{.}\PY{n}{make\PYZus{}spectrum}\PY{p}{(}\PY{p}{)}

\PY{n}{spectrum}\PY{o}{.}\PY{n}{plot\PYZus{}power}\PY{p}{(}\PY{n}{alpha}\PY{o}{=}\PY{l+m+mf}{0.5}\PY{p}{)}
\PY{n}{spectrum2}\PY{o}{.}\PY{n}{plot\PYZus{}power}\PY{p}{(}\PY{n}{alpha}\PY{o}{=}\PY{l+m+mf}{0.5}\PY{p}{)}
\PY{n}{decorate}\PY{p}{(}\PY{n}{xlabel}\PY{o}{=}\PY{l+s+s1}{\PYZsq{}}\PY{l+s+s1}{Frequency (Hz)}\PY{l+s+s1}{\PYZsq{}}\PY{p}{,}
         \PY{n}{ylabel}\PY{o}{=}\PY{l+s+s1}{\PYZsq{}}\PY{l+s+s1}{Power}\PY{l+s+s1}{\PYZsq{}}\PY{p}{)}
\end{Verbatim}
\end{tcolorbox}

    \begin{center}
    \adjustimage{max size={0.9\linewidth}{0.9\paperheight}}{output_15_0.png}
    \end{center}
    { \hspace*{\fill} \\}
    
    Here they are again, plotting power on a log-log scale.

    \begin{tcolorbox}[breakable, size=fbox, boxrule=1pt, pad at break*=1mm,colback=cellbackground, colframe=cellborder]
\prompt{In}{incolor}{10}{\boxspacing}
\begin{Verbatim}[commandchars=\\\{\}]
\PY{n}{spectrum}\PY{o}{.}\PY{n}{plot\PYZus{}power}\PY{p}{(}\PY{n}{alpha}\PY{o}{=}\PY{l+m+mf}{0.5}\PY{p}{)}
\PY{n}{spectrum2}\PY{o}{.}\PY{n}{plot\PYZus{}power}\PY{p}{(}\PY{n}{alpha}\PY{o}{=}\PY{l+m+mf}{0.5}\PY{p}{)}
\PY{n}{decorate}\PY{p}{(}\PY{n}{xlabel}\PY{o}{=}\PY{l+s+s1}{\PYZsq{}}\PY{l+s+s1}{Frequency (Hz)}\PY{l+s+s1}{\PYZsq{}}\PY{p}{,}
         \PY{n}{ylabel}\PY{o}{=}\PY{l+s+s1}{\PYZsq{}}\PY{l+s+s1}{Power}\PY{l+s+s1}{\PYZsq{}}\PY{p}{,}
         \PY{o}{*}\PY{o}{*}\PY{n}{loglog}\PY{p}{)}
\end{Verbatim}
\end{tcolorbox}

    \begin{center}
    \adjustimage{max size={0.9\linewidth}{0.9\paperheight}}{output_17_0.png}
    \end{center}
    { \hspace*{\fill} \\}
    
    So the structure seems to be consistent over time.

We can also look at a spectrogram:

    \begin{tcolorbox}[breakable, size=fbox, boxrule=1pt, pad at break*=1mm,colback=cellbackground, colframe=cellborder]
\prompt{In}{incolor}{11}{\boxspacing}
\begin{Verbatim}[commandchars=\\\{\}]
\PY{n}{segment}\PY{o}{.}\PY{n}{make\PYZus{}spectrogram}\PY{p}{(}\PY{l+m+mi}{512}\PY{p}{)}\PY{o}{.}\PY{n}{plot}\PY{p}{(}\PY{n}{high}\PY{o}{=}\PY{l+m+mi}{5000}\PY{p}{)}
\PY{n}{decorate}\PY{p}{(}\PY{n}{xlabel}\PY{o}{=}\PY{l+s+s1}{\PYZsq{}}\PY{l+s+s1}{Time(s)}\PY{l+s+s1}{\PYZsq{}}\PY{p}{,} \PY{n}{ylabel}\PY{o}{=}\PY{l+s+s1}{\PYZsq{}}\PY{l+s+s1}{Frequency (Hz)}\PY{l+s+s1}{\PYZsq{}}\PY{p}{)}
\end{Verbatim}
\end{tcolorbox}

    \begin{center}
    \adjustimage{max size={0.9\linewidth}{0.9\paperheight}}{output_19_0.png}
    \end{center}
    { \hspace*{\fill} \\}
    
    Within this segment, the overall amplitude drops off, but the mixture of
frequencies seems consistent.

    \textbf{Exercise:} In a noise signal, the mixture of frequencies changes
over time. In the long run, we expect the power at all frequencies to be
equal, but in any sample, the power at each frequency is random.

To estimate the long-term average power at each frequency, we can break
a long signal into segments, compute the power spectrum for each
segment, and then compute the average across the segments. You can read
more about this algorithm at
http://en.wikipedia.org/wiki/Bartlett's\_method.

Implement Bartlett's method and use it to estimate the power spectrum
for a noise wave. Hint: look at the implementation of
\texttt{make\_spectrogram}.

    \begin{tcolorbox}[breakable, size=fbox, boxrule=1pt, pad at break*=1mm,colback=cellbackground, colframe=cellborder]
\prompt{In}{incolor}{12}{\boxspacing}
\begin{Verbatim}[commandchars=\\\{\}]
\PY{k+kn}{from} \PY{n+nn}{thinkdsp} \PY{k+kn}{import} \PY{n}{Spectrum}

\PY{k}{def} \PY{n+nf}{bartlett\PYZus{}method}\PY{p}{(}\PY{n}{wave}\PY{p}{,} \PY{n}{seg\PYZus{}length}\PY{o}{=}\PY{l+m+mi}{512}\PY{p}{,} \PY{n}{win\PYZus{}flag}\PY{o}{=}\PY{k+kc}{True}\PY{p}{)}\PY{p}{:}
    \PY{l+s+sd}{\PYZdq{}\PYZdq{}\PYZdq{}Estimates the power spectrum of a noise wave.}
\PY{l+s+sd}{    }
\PY{l+s+sd}{    wave: Wave}
\PY{l+s+sd}{    seg\PYZus{}length: segment length}
\PY{l+s+sd}{    \PYZdq{}\PYZdq{}\PYZdq{}}
    \PY{c+c1}{\PYZsh{} make a spectrogram and extract the spectrums}
    \PY{n}{spectro} \PY{o}{=} \PY{n}{wave}\PY{o}{.}\PY{n}{make\PYZus{}spectrogram}\PY{p}{(}\PY{n}{seg\PYZus{}length}\PY{p}{,} \PY{n}{win\PYZus{}flag}\PY{p}{)}
    \PY{n}{spectrums} \PY{o}{=} \PY{n}{spectro}\PY{o}{.}\PY{n}{spec\PYZus{}map}\PY{o}{.}\PY{n}{values}\PY{p}{(}\PY{p}{)}
    
    \PY{c+c1}{\PYZsh{} extract the power array from each spectrum}
    \PY{n}{psds} \PY{o}{=} \PY{p}{[}\PY{n}{spectrum}\PY{o}{.}\PY{n}{power} \PY{k}{for} \PY{n}{spectrum} \PY{o+ow}{in} \PY{n}{spectrums}\PY{p}{]}
    
    \PY{c+c1}{\PYZsh{} compute the root mean power (which is like an amplitude)}
    \PY{n}{hs} \PY{o}{=} \PY{n}{np}\PY{o}{.}\PY{n}{sqrt}\PY{p}{(}\PY{n+nb}{sum}\PY{p}{(}\PY{n}{psds}\PY{p}{)} \PY{o}{/} \PY{n+nb}{len}\PY{p}{(}\PY{n}{psds}\PY{p}{)}\PY{p}{)}
    \PY{n}{fs} \PY{o}{=} \PY{n+nb}{next}\PY{p}{(}\PY{n+nb}{iter}\PY{p}{(}\PY{n}{spectrums}\PY{p}{)}\PY{p}{)}\PY{o}{.}\PY{n}{fs}
    
    \PY{c+c1}{\PYZsh{} make a Spectrum with the mean amplitudes}
    \PY{n}{spectrum} \PY{o}{=} \PY{n}{Spectrum}\PY{p}{(}\PY{n}{hs}\PY{p}{,} \PY{n}{fs}\PY{p}{,} \PY{n}{wave}\PY{o}{.}\PY{n}{framerate}\PY{p}{)}
    \PY{k}{return} \PY{n}{spectrum}
\end{Verbatim}
\end{tcolorbox}

    \texttt{bartlett\_method} makes a spectrogram and extracts
\texttt{spec\_map}, which maps from times to Spectrum objects. It
computes the PSD for each spectrum, adds them up, and puts the results
into a Spectrum object.

    \begin{tcolorbox}[breakable, size=fbox, boxrule=1pt, pad at break*=1mm,colback=cellbackground, colframe=cellborder]
\prompt{In}{incolor}{13}{\boxspacing}
\begin{Verbatim}[commandchars=\\\{\}]
\PY{n}{psd} \PY{o}{=} \PY{n}{bartlett\PYZus{}method}\PY{p}{(}\PY{n}{segment}\PY{p}{)}
\PY{n}{psd2} \PY{o}{=} \PY{n}{bartlett\PYZus{}method}\PY{p}{(}\PY{n}{segment2}\PY{p}{)}

\PY{n}{psd}\PY{o}{.}\PY{n}{plot\PYZus{}power}\PY{p}{(}\PY{p}{)}
\PY{n}{psd2}\PY{o}{.}\PY{n}{plot\PYZus{}power}\PY{p}{(}\PY{p}{)}

\PY{n}{decorate}\PY{p}{(}\PY{n}{xlabel}\PY{o}{=}\PY{l+s+s1}{\PYZsq{}}\PY{l+s+s1}{Frequency (Hz)}\PY{l+s+s1}{\PYZsq{}}\PY{p}{,} 
         \PY{n}{ylabel}\PY{o}{=}\PY{l+s+s1}{\PYZsq{}}\PY{l+s+s1}{Power}\PY{l+s+s1}{\PYZsq{}}\PY{p}{,} 
         \PY{o}{*}\PY{o}{*}\PY{n}{loglog}\PY{p}{)}
\end{Verbatim}
\end{tcolorbox}

    \begin{center}
    \adjustimage{max size={0.9\linewidth}{0.9\paperheight}}{output_24_0.png}
    \end{center}
    { \hspace*{\fill} \\}
    
    Now we can see the relationship between power and frequency more
clearly. It is not a simple linear relationship, but it is consistent
across different segments, even in details like the notches near 2000
Hz.

    \hypertarget{exercise-2}{%
\subsection{Exercise 2}\label{exercise-2}}

At \href{https://www.coindesk.com/price/bitcoin}{coindesk} you can
download the daily price of a BitCoin as a CSV file. Read this file and
compute the spectrum of BitCoin prices as a function of time. Does it
resemble white, pink, or Brownian noise?

    \begin{tcolorbox}[breakable, size=fbox, boxrule=1pt, pad at break*=1mm,colback=cellbackground, colframe=cellborder]
\prompt{In}{incolor}{14}{\boxspacing}
\begin{Verbatim}[commandchars=\\\{\}]
\PY{k}{if} \PY{o+ow}{not} \PY{n}{os}\PY{o}{.}\PY{n}{path}\PY{o}{.}\PY{n}{exists}\PY{p}{(}\PY{l+s+s1}{\PYZsq{}}\PY{l+s+s1}{Sounds/BTC\PYZus{}USD\PYZus{}2020\PYZhy{}12\PYZhy{}31\PYZus{}2021\PYZhy{}03\PYZhy{}30\PYZhy{}CoinDesk.csv}\PY{l+s+s1}{\PYZsq{}}\PY{p}{)}\PY{p}{:}
    \PY{o}{!}wget https://github.com/AllenDowney/ThinkDSP/raw/master/code/BTC\PYZus{}USD\PYZus{}2013\PYZhy{}10\PYZhy{}01\PYZus{}2020\PYZhy{}03\PYZhy{}26\PYZhy{}CoinDesk.csv
\end{Verbatim}
\end{tcolorbox}

    \begin{Verbatim}[commandchars=\\\{\}, frame=single, framerule=2mm, rulecolor=\color{outerrorbackground}]
\textcolor{ansi-red-intense}{\textbf{---------------------------------------------------------------------------}}
\textcolor{ansi-red-intense}{\textbf{NameError}}                                 Traceback (most recent call last)
\textcolor{ansi-green-intense}{\textbf{<ipython-input-14-a1f7b6c35127>}} in \textcolor{ansi-cyan}{<module>}
\textcolor{ansi-green-intense}{\textbf{----> 1}}\textcolor{ansi-yellow-intense}{\textbf{ }}\textcolor{ansi-green-intense}{\textbf{if}} \textcolor{ansi-green-intense}{\textbf{not}} os\textcolor{ansi-yellow-intense}{\textbf{.}}path\textcolor{ansi-yellow-intense}{\textbf{.}}exists\textcolor{ansi-yellow-intense}{\textbf{(}}\textcolor{ansi-blue-intense}{\textbf{'Sounds/BTC\_USD\_2020-12-31\_2021-03-30-CoinDesk.csv'}}\textcolor{ansi-yellow-intense}{\textbf{)}}\textcolor{ansi-yellow-intense}{\textbf{:}}
\textcolor{ansi-green}{      2}     get\_ipython\textcolor{ansi-yellow-intense}{\textbf{(}}\textcolor{ansi-yellow-intense}{\textbf{)}}\textcolor{ansi-yellow-intense}{\textbf{.}}system\textcolor{ansi-yellow-intense}{\textbf{(}}\textcolor{ansi-blue-intense}{\textbf{'wget https://github.com/AllenDowney/ThinkDSP/raw/master/code/BTC\_USD\_2013-10-01\_2020-03-26-CoinDesk.csv'}}\textcolor{ansi-yellow-intense}{\textbf{)}}

\textcolor{ansi-red-intense}{\textbf{NameError}}: name 'os' is not defined
    \end{Verbatim}

    \begin{tcolorbox}[breakable, size=fbox, boxrule=1pt, pad at break*=1mm,colback=cellbackground, colframe=cellborder]
\prompt{In}{incolor}{15}{\boxspacing}
\begin{Verbatim}[commandchars=\\\{\}]
\PY{k+kn}{import} \PY{n+nn}{pandas} \PY{k}{as} \PY{n+nn}{pd}

\PY{n}{df} \PY{o}{=} \PY{n}{pd}\PY{o}{.}\PY{n}{read\PYZus{}csv}\PY{p}{(}\PY{l+s+s1}{\PYZsq{}}\PY{l+s+s1}{Sounds/BTC\PYZus{}USD\PYZus{}2020\PYZhy{}12\PYZhy{}31\PYZus{}2021\PYZhy{}03\PYZhy{}30\PYZhy{}CoinDesk.csv}\PY{l+s+s1}{\PYZsq{}}\PY{p}{,} 
                 \PY{n}{parse\PYZus{}dates}\PY{o}{=}\PY{p}{[}\PY{l+m+mi}{0}\PY{p}{]}\PY{p}{)}
\PY{n}{df}
\end{Verbatim}
\end{tcolorbox}

            \begin{tcolorbox}[breakable, size=fbox, boxrule=.5pt, pad at break*=1mm, opacityfill=0]
\prompt{Out}{outcolor}{15}{\boxspacing}
\begin{Verbatim}[commandchars=\\\{\}]
   Currency        Date  Closing Price (USD)  24h Open (USD)  24h High (USD)  \textbackslash{}
0       BTC  2020-12-31         28768.836208    27349.327233    28928.214391
1       BTC  2021-01-01         29111.521567    28872.829775    29280.045328
2       BTC  2021-01-02         29333.605121    28935.810981    29601.594898
3       BTC  2021-01-03         32154.167363    29353.640608    33064.673534
4       BTC  2021-01-04         33002.536427    32074.106611    34452.080337
..      {\ldots}         {\ldots}                  {\ldots}             {\ldots}             {\ldots}
85      BTC  2021-03-26         52173.867980    52335.565034    53209.406384
86      BTC  2021-03-27         54483.045732    51344.048980    54806.881514
87      BTC  2021-03-28         56234.356105    55067.824399    56520.307287
88      BTC  2021-03-29         55343.925815    55843.748916    56541.006527
89      BTC  2021-03-30         57627.679249    55786.546075    58353.529393

    24h Low (USD)
0    27349.283204
1    27916.625059
2    28753.412314
3    29012.927887
4    31885.581619
..            {\ldots}
85   50458.099965
86   51286.291177
87   54022.076941
88   54741.626373
89   54929.680588

[90 rows x 6 columns]
\end{Verbatim}
\end{tcolorbox}
        
    \begin{tcolorbox}[breakable, size=fbox, boxrule=1pt, pad at break*=1mm,colback=cellbackground, colframe=cellborder]
\prompt{In}{incolor}{16}{\boxspacing}
\begin{Verbatim}[commandchars=\\\{\}]
\PY{n}{ys} \PY{o}{=} \PY{n}{df}\PY{p}{[}\PY{l+s+s1}{\PYZsq{}}\PY{l+s+s1}{Closing Price (USD)}\PY{l+s+s1}{\PYZsq{}}\PY{p}{]}
\PY{n}{ts} \PY{o}{=} \PY{n}{df}\PY{o}{.}\PY{n}{index}
\end{Verbatim}
\end{tcolorbox}

    \begin{tcolorbox}[breakable, size=fbox, boxrule=1pt, pad at break*=1mm,colback=cellbackground, colframe=cellborder]
\prompt{In}{incolor}{17}{\boxspacing}
\begin{Verbatim}[commandchars=\\\{\}]
\PY{k+kn}{from} \PY{n+nn}{thinkdsp} \PY{k+kn}{import} \PY{n}{Wave}

\PY{n}{wave} \PY{o}{=} \PY{n}{Wave}\PY{p}{(}\PY{n}{ys}\PY{p}{,} \PY{n}{ts}\PY{p}{,} \PY{n}{framerate}\PY{o}{=}\PY{l+m+mi}{1}\PY{p}{)}
\PY{n}{wave}\PY{o}{.}\PY{n}{plot}\PY{p}{(}\PY{p}{)}
\PY{n}{decorate}\PY{p}{(}\PY{n}{xlabel}\PY{o}{=}\PY{l+s+s1}{\PYZsq{}}\PY{l+s+s1}{Time (days)}\PY{l+s+s1}{\PYZsq{}}\PY{p}{)}
\end{Verbatim}
\end{tcolorbox}

    \begin{center}
    \adjustimage{max size={0.9\linewidth}{0.9\paperheight}}{output_30_0.png}
    \end{center}
    { \hspace*{\fill} \\}
    
    \begin{tcolorbox}[breakable, size=fbox, boxrule=1pt, pad at break*=1mm,colback=cellbackground, colframe=cellborder]
\prompt{In}{incolor}{18}{\boxspacing}
\begin{Verbatim}[commandchars=\\\{\}]
\PY{n}{spectrum} \PY{o}{=} \PY{n}{wave}\PY{o}{.}\PY{n}{make\PYZus{}spectrum}\PY{p}{(}\PY{p}{)}
\PY{n}{spectrum}\PY{o}{.}\PY{n}{plot\PYZus{}power}\PY{p}{(}\PY{p}{)}
\PY{n}{decorate}\PY{p}{(}\PY{n}{xlabel}\PY{o}{=}\PY{l+s+s1}{\PYZsq{}}\PY{l+s+s1}{Frequency (1/days)}\PY{l+s+s1}{\PYZsq{}}\PY{p}{,}
         \PY{n}{ylabel}\PY{o}{=}\PY{l+s+s1}{\PYZsq{}}\PY{l+s+s1}{Power}\PY{l+s+s1}{\PYZsq{}}\PY{p}{,} 
         \PY{o}{*}\PY{o}{*}\PY{n}{loglog}\PY{p}{)}
\end{Verbatim}
\end{tcolorbox}

    \begin{center}
    \adjustimage{max size={0.9\linewidth}{0.9\paperheight}}{output_31_0.png}
    \end{center}
    { \hspace*{\fill} \\}
    
    \begin{tcolorbox}[breakable, size=fbox, boxrule=1pt, pad at break*=1mm,colback=cellbackground, colframe=cellborder]
\prompt{In}{incolor}{19}{\boxspacing}
\begin{Verbatim}[commandchars=\\\{\}]
\PY{n}{spectrum}\PY{o}{.}\PY{n}{estimate\PYZus{}slope}\PY{p}{(}\PY{p}{)}\PY{p}{[}\PY{l+m+mi}{0}\PY{p}{]}
\end{Verbatim}
\end{tcolorbox}

            \begin{tcolorbox}[breakable, size=fbox, boxrule=.5pt, pad at break*=1mm, opacityfill=0]
\prompt{Out}{outcolor}{19}{\boxspacing}
\begin{Verbatim}[commandchars=\\\{\}]
-1.6128256161954575
\end{Verbatim}
\end{tcolorbox}
        
    Red noise should have a slope of -2. The slope of this PSD is close to
1.6, so it's hard to say if we should consider it red noise or if we
should say it's a kind of pink noise.

    \hypertarget{exercise-3}{%
\subsection{Exercise 3}\label{exercise-3}}

A Geiger counter is a device that detects radiation. When an ionizing
particle strikes the detector, it outputs a surge of current. The total
output at a point in time can be modeled as uncorrelated Poisson (UP)
noise, where each sample is a random quantity from a Poisson
distribution, which corresponds to the number of particles detected
during an interval.

Write a class called \texttt{UncorrelatedPoissonNoise} that inherits
from \texttt{\_Noise} and provides \texttt{evaluate}. It should use
\texttt{np.random.poisson} to generate random values from a Poisson
distribution. The parameter of this function, \texttt{lam}, is the
average number of particles during each interval. You can use the
attribute \texttt{amp} to specify \texttt{lam}. For example, if the
framerate is 10 kHz and \texttt{amp} is 0.001, we expect about 10
``clicks'' per second.

Generate about a second of UP noise and listen to it. For low values of
\texttt{amp}, like 0.001, it should sound like a Geiger counter. For
higher values it should sound like white noise. Compute and plot the
power spectrum to see whether it looks like white noise.

    \begin{tcolorbox}[breakable, size=fbox, boxrule=1pt, pad at break*=1mm,colback=cellbackground, colframe=cellborder]
\prompt{In}{incolor}{20}{\boxspacing}
\begin{Verbatim}[commandchars=\\\{\}]
\PY{k+kn}{from} \PY{n+nn}{thinkdsp} \PY{k+kn}{import} \PY{n}{Noise}

\PY{k}{class} \PY{n+nc}{UncorrelatedPoissonNoise}\PY{p}{(}\PY{n}{Noise}\PY{p}{)}\PY{p}{:}
    \PY{l+s+sd}{\PYZdq{}\PYZdq{}\PYZdq{}Represents uncorrelated Poisson noise.\PYZdq{}\PYZdq{}\PYZdq{}}

    \PY{k}{def} \PY{n+nf}{evaluate}\PY{p}{(}\PY{n+nb+bp}{self}\PY{p}{,} \PY{n}{ts}\PY{p}{)}\PY{p}{:}
        \PY{l+s+sd}{\PYZdq{}\PYZdq{}\PYZdq{}Evaluates the signal at the given times.}

\PY{l+s+sd}{        ts: float array of times}
\PY{l+s+sd}{        }
\PY{l+s+sd}{        returns: float wave array}
\PY{l+s+sd}{        \PYZdq{}\PYZdq{}\PYZdq{}}
        \PY{n}{ys} \PY{o}{=} \PY{n}{np}\PY{o}{.}\PY{n}{random}\PY{o}{.}\PY{n}{poisson}\PY{p}{(}\PY{n+nb+bp}{self}\PY{o}{.}\PY{n}{amp}\PY{p}{,} \PY{n+nb}{len}\PY{p}{(}\PY{n}{ts}\PY{p}{)}\PY{p}{)}
        \PY{k}{return} \PY{n}{ys}
\end{Verbatim}
\end{tcolorbox}

    Here's what it sounds like at low levels of ``radiation''.

    \begin{tcolorbox}[breakable, size=fbox, boxrule=1pt, pad at break*=1mm,colback=cellbackground, colframe=cellborder]
\prompt{In}{incolor}{21}{\boxspacing}
\begin{Verbatim}[commandchars=\\\{\}]
\PY{n}{amp} \PY{o}{=} \PY{l+m+mf}{0.001}
\PY{n}{framerate} \PY{o}{=} \PY{l+m+mi}{10000}
\PY{n}{duration} \PY{o}{=} \PY{l+m+mi}{1}

\PY{n}{signal} \PY{o}{=} \PY{n}{UncorrelatedPoissonNoise}\PY{p}{(}\PY{n}{amp}\PY{o}{=}\PY{n}{amp}\PY{p}{)}
\PY{n}{wave} \PY{o}{=} \PY{n}{signal}\PY{o}{.}\PY{n}{make\PYZus{}wave}\PY{p}{(}\PY{n}{duration}\PY{o}{=}\PY{n}{duration}\PY{p}{,} \PY{n}{framerate}\PY{o}{=}\PY{n}{framerate}\PY{p}{)}
\PY{n}{wave}\PY{o}{.}\PY{n}{make\PYZus{}audio}\PY{p}{(}\PY{p}{)}
\end{Verbatim}
\end{tcolorbox}

            \begin{tcolorbox}[breakable, size=fbox, boxrule=.5pt, pad at break*=1mm, opacityfill=0]
\prompt{Out}{outcolor}{21}{\boxspacing}
\begin{Verbatim}[commandchars=\\\{\}]
<IPython.lib.display.Audio object>
\end{Verbatim}
\end{tcolorbox}
        
    To check that things worked, we compare the expected number of particles
and the actual number:

    \begin{tcolorbox}[breakable, size=fbox, boxrule=1pt, pad at break*=1mm,colback=cellbackground, colframe=cellborder]
\prompt{In}{incolor}{22}{\boxspacing}
\begin{Verbatim}[commandchars=\\\{\}]
\PY{n}{expected} \PY{o}{=} \PY{n}{amp} \PY{o}{*} \PY{n}{framerate} \PY{o}{*} \PY{n}{duration}
\PY{n}{actual} \PY{o}{=} \PY{n+nb}{sum}\PY{p}{(}\PY{n}{wave}\PY{o}{.}\PY{n}{ys}\PY{p}{)}
\PY{n+nb}{print}\PY{p}{(}\PY{n}{expected}\PY{p}{,} \PY{n}{actual}\PY{p}{)}
\end{Verbatim}
\end{tcolorbox}

    \begin{Verbatim}[commandchars=\\\{\}]
10.0 17
    \end{Verbatim}

    Here's what the wave looks like:

    \begin{tcolorbox}[breakable, size=fbox, boxrule=1pt, pad at break*=1mm,colback=cellbackground, colframe=cellborder]
\prompt{In}{incolor}{23}{\boxspacing}
\begin{Verbatim}[commandchars=\\\{\}]
\PY{n}{wave}\PY{o}{.}\PY{n}{plot}\PY{p}{(}\PY{p}{)}
\end{Verbatim}
\end{tcolorbox}

    \begin{center}
    \adjustimage{max size={0.9\linewidth}{0.9\paperheight}}{output_41_0.png}
    \end{center}
    { \hspace*{\fill} \\}
    
    And here's its power spectrum on a log-log scale.

    \begin{tcolorbox}[breakable, size=fbox, boxrule=1pt, pad at break*=1mm,colback=cellbackground, colframe=cellborder]
\prompt{In}{incolor}{24}{\boxspacing}
\begin{Verbatim}[commandchars=\\\{\}]
\PY{n}{spectrum} \PY{o}{=} \PY{n}{wave}\PY{o}{.}\PY{n}{make\PYZus{}spectrum}\PY{p}{(}\PY{p}{)}
\PY{n}{spectrum}\PY{o}{.}\PY{n}{plot\PYZus{}power}\PY{p}{(}\PY{p}{)}
\PY{n}{decorate}\PY{p}{(}\PY{n}{xlabel}\PY{o}{=}\PY{l+s+s1}{\PYZsq{}}\PY{l+s+s1}{Frequency (Hz)}\PY{l+s+s1}{\PYZsq{}}\PY{p}{,}
         \PY{n}{ylabel}\PY{o}{=}\PY{l+s+s1}{\PYZsq{}}\PY{l+s+s1}{Power}\PY{l+s+s1}{\PYZsq{}}\PY{p}{,}
         \PY{o}{*}\PY{o}{*}\PY{n}{loglog}\PY{p}{)}
\end{Verbatim}
\end{tcolorbox}

    \begin{center}
    \adjustimage{max size={0.9\linewidth}{0.9\paperheight}}{output_43_0.png}
    \end{center}
    { \hspace*{\fill} \\}
    
    Looks like white noise, and the slope is close to 0.

    \begin{tcolorbox}[breakable, size=fbox, boxrule=1pt, pad at break*=1mm,colback=cellbackground, colframe=cellborder]
\prompt{In}{incolor}{25}{\boxspacing}
\begin{Verbatim}[commandchars=\\\{\}]
\PY{n}{spectrum}\PY{o}{.}\PY{n}{estimate\PYZus{}slope}\PY{p}{(}\PY{p}{)}\PY{o}{.}\PY{n}{slope}
\end{Verbatim}
\end{tcolorbox}

            \begin{tcolorbox}[breakable, size=fbox, boxrule=.5pt, pad at break*=1mm, opacityfill=0]
\prompt{Out}{outcolor}{25}{\boxspacing}
\begin{Verbatim}[commandchars=\\\{\}]
-0.016770072697198667
\end{Verbatim}
\end{tcolorbox}
        
    With a higher arrival rate, it sounds more like white noise:

    \begin{tcolorbox}[breakable, size=fbox, boxrule=1pt, pad at break*=1mm,colback=cellbackground, colframe=cellborder]
\prompt{In}{incolor}{26}{\boxspacing}
\begin{Verbatim}[commandchars=\\\{\}]
\PY{n}{amp} \PY{o}{=} \PY{l+m+mf}{0.5}
\PY{n}{framerate} \PY{o}{=} \PY{l+m+mi}{10000}
\PY{n}{duration} \PY{o}{=} \PY{l+m+mi}{1}

\PY{n}{signal} \PY{o}{=} \PY{n}{UncorrelatedPoissonNoise}\PY{p}{(}\PY{n}{amp}\PY{o}{=}\PY{n}{amp}\PY{p}{)}
\PY{n}{wave} \PY{o}{=} \PY{n}{signal}\PY{o}{.}\PY{n}{make\PYZus{}wave}\PY{p}{(}\PY{n}{duration}\PY{o}{=}\PY{n}{duration}\PY{p}{,} \PY{n}{framerate}\PY{o}{=}\PY{n}{framerate}\PY{p}{)}
\PY{n}{wave}\PY{o}{.}\PY{n}{make\PYZus{}audio}\PY{p}{(}\PY{p}{)}
\end{Verbatim}
\end{tcolorbox}

            \begin{tcolorbox}[breakable, size=fbox, boxrule=.5pt, pad at break*=1mm, opacityfill=0]
\prompt{Out}{outcolor}{26}{\boxspacing}
\begin{Verbatim}[commandchars=\\\{\}]
<IPython.lib.display.Audio object>
\end{Verbatim}
\end{tcolorbox}
        
    It looks more like a signal:

    \begin{tcolorbox}[breakable, size=fbox, boxrule=1pt, pad at break*=1mm,colback=cellbackground, colframe=cellborder]
\prompt{In}{incolor}{27}{\boxspacing}
\begin{Verbatim}[commandchars=\\\{\}]
\PY{n}{wave}\PY{o}{.}\PY{n}{plot}\PY{p}{(}\PY{p}{)}
\end{Verbatim}
\end{tcolorbox}

    \begin{center}
    \adjustimage{max size={0.9\linewidth}{0.9\paperheight}}{output_49_0.png}
    \end{center}
    { \hspace*{\fill} \\}
    
    And the spectrum converges on Gaussian noise.

    \begin{tcolorbox}[breakable, size=fbox, boxrule=1pt, pad at break*=1mm,colback=cellbackground, colframe=cellborder]
\prompt{In}{incolor}{28}{\boxspacing}
\begin{Verbatim}[commandchars=\\\{\}]
\PY{k+kn}{import} \PY{n+nn}{matplotlib}\PY{n+nn}{.}\PY{n+nn}{pyplot} \PY{k}{as} \PY{n+nn}{plt}

\PY{k}{def} \PY{n+nf}{normal\PYZus{}prob\PYZus{}plot}\PY{p}{(}\PY{n}{sample}\PY{p}{,} \PY{n}{fit\PYZus{}color}\PY{o}{=}\PY{l+s+s1}{\PYZsq{}}\PY{l+s+s1}{0.8}\PY{l+s+s1}{\PYZsq{}}\PY{p}{,} \PY{o}{*}\PY{o}{*}\PY{n}{options}\PY{p}{)}\PY{p}{:}
    \PY{l+s+sd}{\PYZdq{}\PYZdq{}\PYZdq{}Makes a normal probability plot with a fitted line.}

\PY{l+s+sd}{    sample: sequence of numbers}
\PY{l+s+sd}{    fit\PYZus{}color: color string for the fitted line}
\PY{l+s+sd}{    options: passed along to Plot}
\PY{l+s+sd}{    \PYZdq{}\PYZdq{}\PYZdq{}}
    \PY{n}{n} \PY{o}{=} \PY{n+nb}{len}\PY{p}{(}\PY{n}{sample}\PY{p}{)}
    \PY{n}{xs} \PY{o}{=} \PY{n}{np}\PY{o}{.}\PY{n}{random}\PY{o}{.}\PY{n}{normal}\PY{p}{(}\PY{l+m+mi}{0}\PY{p}{,} \PY{l+m+mi}{1}\PY{p}{,} \PY{n}{n}\PY{p}{)}
    \PY{n}{xs}\PY{o}{.}\PY{n}{sort}\PY{p}{(}\PY{p}{)}
    
    \PY{n}{ys} \PY{o}{=} \PY{n}{np}\PY{o}{.}\PY{n}{sort}\PY{p}{(}\PY{n}{sample}\PY{p}{)}
    
    \PY{n}{mean}\PY{p}{,} \PY{n}{std} \PY{o}{=} \PY{n}{np}\PY{o}{.}\PY{n}{mean}\PY{p}{(}\PY{n}{sample}\PY{p}{)}\PY{p}{,} \PY{n}{np}\PY{o}{.}\PY{n}{std}\PY{p}{(}\PY{n}{sample}\PY{p}{)}
    \PY{n}{fit\PYZus{}ys} \PY{o}{=} \PY{n}{mean} \PY{o}{+} \PY{n}{std} \PY{o}{*} \PY{n}{xs}
    \PY{n}{plt}\PY{o}{.}\PY{n}{plot}\PY{p}{(}\PY{n}{xs}\PY{p}{,} \PY{n}{fit\PYZus{}ys}\PY{p}{,} \PY{n}{color}\PY{o}{=}\PY{l+s+s1}{\PYZsq{}}\PY{l+s+s1}{gray}\PY{l+s+s1}{\PYZsq{}}\PY{p}{,} \PY{n}{alpha}\PY{o}{=}\PY{l+m+mf}{0.5}\PY{p}{,} \PY{n}{label}\PY{o}{=}\PY{l+s+s1}{\PYZsq{}}\PY{l+s+s1}{model}\PY{l+s+s1}{\PYZsq{}}\PY{p}{)}

    \PY{n}{plt}\PY{o}{.}\PY{n}{plot}\PY{p}{(}\PY{n}{xs}\PY{p}{,} \PY{n}{ys}\PY{p}{,} \PY{o}{*}\PY{o}{*}\PY{n}{options}\PY{p}{)}
\end{Verbatim}
\end{tcolorbox}

    \begin{tcolorbox}[breakable, size=fbox, boxrule=1pt, pad at break*=1mm,colback=cellbackground, colframe=cellborder]
\prompt{In}{incolor}{29}{\boxspacing}
\begin{Verbatim}[commandchars=\\\{\}]
\PY{n}{spectrum} \PY{o}{=} \PY{n}{wave}\PY{o}{.}\PY{n}{make\PYZus{}spectrum}\PY{p}{(}\PY{p}{)}
\PY{n}{spectrum}\PY{o}{.}\PY{n}{hs}\PY{p}{[}\PY{l+m+mi}{0}\PY{p}{]} \PY{o}{=} \PY{l+m+mi}{0}

\PY{n}{normal\PYZus{}prob\PYZus{}plot}\PY{p}{(}\PY{n}{spectrum}\PY{o}{.}\PY{n}{real}\PY{p}{,} \PY{n}{label}\PY{o}{=}\PY{l+s+s1}{\PYZsq{}}\PY{l+s+s1}{real}\PY{l+s+s1}{\PYZsq{}}\PY{p}{)}
\PY{n}{decorate}\PY{p}{(}\PY{n}{xlabel}\PY{o}{=}\PY{l+s+s1}{\PYZsq{}}\PY{l+s+s1}{Normal sample}\PY{l+s+s1}{\PYZsq{}}\PY{p}{,}
        \PY{n}{ylabel}\PY{o}{=}\PY{l+s+s1}{\PYZsq{}}\PY{l+s+s1}{Power}\PY{l+s+s1}{\PYZsq{}}\PY{p}{)}
\end{Verbatim}
\end{tcolorbox}

    \begin{center}
    \adjustimage{max size={0.9\linewidth}{0.9\paperheight}}{output_52_0.png}
    \end{center}
    { \hspace*{\fill} \\}
    
    \begin{tcolorbox}[breakable, size=fbox, boxrule=1pt, pad at break*=1mm,colback=cellbackground, colframe=cellborder]
\prompt{In}{incolor}{30}{\boxspacing}
\begin{Verbatim}[commandchars=\\\{\}]
\PY{n}{normal\PYZus{}prob\PYZus{}plot}\PY{p}{(}\PY{n}{spectrum}\PY{o}{.}\PY{n}{imag}\PY{p}{,} \PY{n}{label}\PY{o}{=}\PY{l+s+s1}{\PYZsq{}}\PY{l+s+s1}{imag}\PY{l+s+s1}{\PYZsq{}}\PY{p}{,} \PY{n}{color}\PY{o}{=}\PY{l+s+s1}{\PYZsq{}}\PY{l+s+s1}{C1}\PY{l+s+s1}{\PYZsq{}}\PY{p}{)}
\PY{n}{decorate}\PY{p}{(}\PY{n}{xlabel}\PY{o}{=}\PY{l+s+s1}{\PYZsq{}}\PY{l+s+s1}{Normal sample}\PY{l+s+s1}{\PYZsq{}}\PY{p}{)}
\end{Verbatim}
\end{tcolorbox}

    \begin{center}
    \adjustimage{max size={0.9\linewidth}{0.9\paperheight}}{output_53_0.png}
    \end{center}
    { \hspace*{\fill} \\}
    
    \hypertarget{exercise-4}{%
\subsection{Exercise 4}\label{exercise-4}}

The algorithm in this chapter for generating pink noise is conceptually
simple but computationally expensive. There are more efficient
alternatives, like the Voss-McCartney algorithm. Research this method,
implement it, compute the spectrum of the result, and confirm that it
has the desired relationship between power and frequency.

    \textbf{Solution:} The fundamental idea of this algorithm is to add up
several sequences of random numbers that get updates at different
sampling rates. The first source should get updated at every time step;
the second source every other time step, the third source ever fourth
step, and so on.

In the original algorithm, the updates are evenly spaced. In an
alternative proposed at http://www.firstpr.com.au/dsp/pink-noise/, they
are randomly spaced.

My implementation starts with an array with one row per timestep and one
column for each of the white noise sources. Initially, the first row and
the first column are random and the rest of the array is Nan.

    \begin{tcolorbox}[breakable, size=fbox, boxrule=1pt, pad at break*=1mm,colback=cellbackground, colframe=cellborder]
\prompt{In}{incolor}{31}{\boxspacing}
\begin{Verbatim}[commandchars=\\\{\}]
\PY{n}{nrows} \PY{o}{=} \PY{l+m+mi}{100}
\PY{n}{ncols} \PY{o}{=} \PY{l+m+mi}{5}

\PY{n}{array} \PY{o}{=} \PY{n}{np}\PY{o}{.}\PY{n}{empty}\PY{p}{(}\PY{p}{(}\PY{n}{nrows}\PY{p}{,} \PY{n}{ncols}\PY{p}{)}\PY{p}{)}
\PY{n}{array}\PY{o}{.}\PY{n}{fill}\PY{p}{(}\PY{n}{np}\PY{o}{.}\PY{n}{nan}\PY{p}{)}
\PY{n}{array}\PY{p}{[}\PY{l+m+mi}{0}\PY{p}{,} \PY{p}{:}\PY{p}{]} \PY{o}{=} \PY{n}{np}\PY{o}{.}\PY{n}{random}\PY{o}{.}\PY{n}{random}\PY{p}{(}\PY{n}{ncols}\PY{p}{)}
\PY{n}{array}\PY{p}{[}\PY{p}{:}\PY{p}{,} \PY{l+m+mi}{0}\PY{p}{]} \PY{o}{=} \PY{n}{np}\PY{o}{.}\PY{n}{random}\PY{o}{.}\PY{n}{random}\PY{p}{(}\PY{n}{nrows}\PY{p}{)}
\PY{n}{array}\PY{p}{[}\PY{l+m+mi}{0}\PY{p}{:}\PY{l+m+mi}{6}\PY{p}{]}
\end{Verbatim}
\end{tcolorbox}

            \begin{tcolorbox}[breakable, size=fbox, boxrule=.5pt, pad at break*=1mm, opacityfill=0]
\prompt{Out}{outcolor}{31}{\boxspacing}
\begin{Verbatim}[commandchars=\\\{\}]
array([[0.67240401, 0.42058616, 0.74990371, 0.14055787, 0.60537631],
       [0.40886731,        nan,        nan,        nan,        nan],
       [0.311841  ,        nan,        nan,        nan,        nan],
       [0.2319267 ,        nan,        nan,        nan,        nan],
       [0.41802469,        nan,        nan,        nan,        nan],
       [0.17323362,        nan,        nan,        nan,        nan]])
\end{Verbatim}
\end{tcolorbox}
        
    The next step is to choose the locations where the random sources
change. If the number of rows is \(n\), the number of changes in the
first column is \(n\), the number in the second column is \(n/2\) on
average, the number in the third column is \(n/4\) on average, etc.

So the total number of changes in the matrix is \(2n\) on average; since
\(n\) of those are in the first column, the other \(n\) are in the rest
of the matrix.

To place the remaining \(n\) changes, we generate random columns from a
geometric distribution with \(p=0.5\). If we generate a value out of
bounds, we set it to 0 (so the first column gets the extras).

    \begin{tcolorbox}[breakable, size=fbox, boxrule=1pt, pad at break*=1mm,colback=cellbackground, colframe=cellborder]
\prompt{In}{incolor}{32}{\boxspacing}
\begin{Verbatim}[commandchars=\\\{\}]
\PY{n}{p} \PY{o}{=} \PY{l+m+mf}{0.5}
\PY{n}{n} \PY{o}{=} \PY{n}{nrows}
\PY{n}{cols} \PY{o}{=} \PY{n}{np}\PY{o}{.}\PY{n}{random}\PY{o}{.}\PY{n}{geometric}\PY{p}{(}\PY{n}{p}\PY{p}{,} \PY{n}{n}\PY{p}{)}
\PY{n}{cols}\PY{p}{[}\PY{n}{cols} \PY{o}{\PYZgt{}}\PY{o}{=} \PY{n}{ncols}\PY{p}{]} \PY{o}{=} \PY{l+m+mi}{0}
\PY{n}{cols}
\end{Verbatim}
\end{tcolorbox}

            \begin{tcolorbox}[breakable, size=fbox, boxrule=.5pt, pad at break*=1mm, opacityfill=0]
\prompt{Out}{outcolor}{32}{\boxspacing}
\begin{Verbatim}[commandchars=\\\{\}]
array([2, 4, 2, 2, 2, 2, 4, 1, 2, 3, 1, 1, 1, 1, 2, 2, 0, 3, 0, 3, 1, 3,
       2, 1, 2, 2, 1, 0, 1, 1, 3, 3, 2, 0, 2, 4, 1, 2, 3, 1, 1, 1, 2, 1,
       1, 4, 4, 2, 2, 1, 2, 3, 1, 1, 0, 1, 0, 1, 1, 0, 2, 3, 1, 1, 0, 1,
       1, 2, 1, 1, 1, 2, 1, 2, 2, 4, 1, 3, 1, 1, 2, 2, 2, 1, 1, 1, 2, 2,
       2, 1, 2, 1, 1, 2, 2, 1, 4, 1, 3, 2])
\end{Verbatim}
\end{tcolorbox}
        
    Within each column, we choose a random row from a uniform distribution.
Ideally we would choose without replacement, but it is faster and easier
to choose with replacement, and I doubt it matters.

    \begin{tcolorbox}[breakable, size=fbox, boxrule=1pt, pad at break*=1mm,colback=cellbackground, colframe=cellborder]
\prompt{In}{incolor}{33}{\boxspacing}
\begin{Verbatim}[commandchars=\\\{\}]
\PY{n}{rows} \PY{o}{=} \PY{n}{np}\PY{o}{.}\PY{n}{random}\PY{o}{.}\PY{n}{randint}\PY{p}{(}\PY{n}{nrows}\PY{p}{,} \PY{n}{size}\PY{o}{=}\PY{n}{n}\PY{p}{)}
\PY{n}{rows}
\end{Verbatim}
\end{tcolorbox}

            \begin{tcolorbox}[breakable, size=fbox, boxrule=.5pt, pad at break*=1mm, opacityfill=0]
\prompt{Out}{outcolor}{33}{\boxspacing}
\begin{Verbatim}[commandchars=\\\{\}]
array([31, 10, 39, 79, 62, 62, 75,  5, 42, 93, 68,  1, 96, 11, 10, 37, 78,
        9, 47, 26, 10, 26,  1,  2, 12, 77, 29, 28, 19, 58,  5, 51, 51, 55,
       15, 10, 62, 41, 96, 47, 90, 82, 33, 15, 97, 41, 88, 20, 60, 28, 70,
       83, 87,  3, 41, 48,  3, 87,  5, 21, 64, 29, 24, 75, 48, 25, 11, 15,
       30, 69, 29, 14, 98, 16, 62, 57, 81, 82, 45, 81, 77, 51, 43, 29, 85,
       87, 75, 41, 32, 27, 82, 13, 17, 97, 83, 62, 37, 63, 19, 70])
\end{Verbatim}
\end{tcolorbox}
        
    Now we can put random values at rach of the change points.

    \begin{tcolorbox}[breakable, size=fbox, boxrule=1pt, pad at break*=1mm,colback=cellbackground, colframe=cellborder]
\prompt{In}{incolor}{34}{\boxspacing}
\begin{Verbatim}[commandchars=\\\{\}]
\PY{n}{array}\PY{p}{[}\PY{n}{rows}\PY{p}{,} \PY{n}{cols}\PY{p}{]} \PY{o}{=} \PY{n}{np}\PY{o}{.}\PY{n}{random}\PY{o}{.}\PY{n}{random}\PY{p}{(}\PY{n}{n}\PY{p}{)}
\PY{n}{array}\PY{p}{[}\PY{l+m+mi}{0}\PY{p}{:}\PY{l+m+mi}{6}\PY{p}{]}
\end{Verbatim}
\end{tcolorbox}

            \begin{tcolorbox}[breakable, size=fbox, boxrule=.5pt, pad at break*=1mm, opacityfill=0]
\prompt{Out}{outcolor}{34}{\boxspacing}
\begin{Verbatim}[commandchars=\\\{\}]
array([[0.67240401, 0.42058616, 0.74990371, 0.14055787, 0.60537631],
       [0.40886731, 0.77466537, 0.71034024,        nan,        nan],
       [0.311841  , 0.23040389,        nan,        nan,        nan],
       [0.00521517, 0.2977681 ,        nan,        nan,        nan],
       [0.41802469,        nan,        nan,        nan,        nan],
       [0.17323362, 0.08568578,        nan, 0.42213265,        nan]])
\end{Verbatim}
\end{tcolorbox}
        
    Next we want to do a zero-order hold to fill in the NaNs. NumPy doesn't
do that, but Pandas does. So I'll create a DataFrame:

    \begin{tcolorbox}[breakable, size=fbox, boxrule=1pt, pad at break*=1mm,colback=cellbackground, colframe=cellborder]
\prompt{In}{incolor}{35}{\boxspacing}
\begin{Verbatim}[commandchars=\\\{\}]
\PY{n}{df} \PY{o}{=} \PY{n}{pd}\PY{o}{.}\PY{n}{DataFrame}\PY{p}{(}\PY{n}{array}\PY{p}{)}
\PY{n}{df}\PY{o}{.}\PY{n}{head}\PY{p}{(}\PY{p}{)}
\end{Verbatim}
\end{tcolorbox}

            \begin{tcolorbox}[breakable, size=fbox, boxrule=.5pt, pad at break*=1mm, opacityfill=0]
\prompt{Out}{outcolor}{35}{\boxspacing}
\begin{Verbatim}[commandchars=\\\{\}]
          0         1         2         3         4
0  0.672404  0.420586  0.749904  0.140558  0.605376
1  0.408867  0.774665  0.710340       NaN       NaN
2  0.311841  0.230404       NaN       NaN       NaN
3  0.005215  0.297768       NaN       NaN       NaN
4  0.418025       NaN       NaN       NaN       NaN
\end{Verbatim}
\end{tcolorbox}
        
    And then use \texttt{fillna} along the columns.

    \begin{tcolorbox}[breakable, size=fbox, boxrule=1pt, pad at break*=1mm,colback=cellbackground, colframe=cellborder]
\prompt{In}{incolor}{36}{\boxspacing}
\begin{Verbatim}[commandchars=\\\{\}]
\PY{n}{filled} \PY{o}{=} \PY{n}{df}\PY{o}{.}\PY{n}{fillna}\PY{p}{(}\PY{n}{method}\PY{o}{=}\PY{l+s+s1}{\PYZsq{}}\PY{l+s+s1}{ffill}\PY{l+s+s1}{\PYZsq{}}\PY{p}{,} \PY{n}{axis}\PY{o}{=}\PY{l+m+mi}{0}\PY{p}{)}
\PY{n}{filled}\PY{o}{.}\PY{n}{head}\PY{p}{(}\PY{p}{)}
\end{Verbatim}
\end{tcolorbox}

            \begin{tcolorbox}[breakable, size=fbox, boxrule=.5pt, pad at break*=1mm, opacityfill=0]
\prompt{Out}{outcolor}{36}{\boxspacing}
\begin{Verbatim}[commandchars=\\\{\}]
          0         1         2         3         4
0  0.672404  0.420586  0.749904  0.140558  0.605376
1  0.408867  0.774665  0.710340  0.140558  0.605376
2  0.311841  0.230404  0.710340  0.140558  0.605376
3  0.005215  0.297768  0.710340  0.140558  0.605376
4  0.418025  0.297768  0.710340  0.140558  0.605376
\end{Verbatim}
\end{tcolorbox}
        
    Finally we add up the rows.

    \begin{tcolorbox}[breakable, size=fbox, boxrule=1pt, pad at break*=1mm,colback=cellbackground, colframe=cellborder]
\prompt{In}{incolor}{37}{\boxspacing}
\begin{Verbatim}[commandchars=\\\{\}]
\PY{n}{total} \PY{o}{=} \PY{n}{filled}\PY{o}{.}\PY{n}{sum}\PY{p}{(}\PY{n}{axis}\PY{o}{=}\PY{l+m+mi}{1}\PY{p}{)}
\PY{n}{total}\PY{o}{.}\PY{n}{head}\PY{p}{(}\PY{p}{)}
\end{Verbatim}
\end{tcolorbox}

            \begin{tcolorbox}[breakable, size=fbox, boxrule=.5pt, pad at break*=1mm, opacityfill=0]
\prompt{Out}{outcolor}{37}{\boxspacing}
\begin{Verbatim}[commandchars=\\\{\}]
0    2.588828
1    2.639807
2    1.998519
3    1.759258
4    2.172067
dtype: float64
\end{Verbatim}
\end{tcolorbox}
        
    If we put the results into a Wave, here's what it looks like:

    \begin{tcolorbox}[breakable, size=fbox, boxrule=1pt, pad at break*=1mm,colback=cellbackground, colframe=cellborder]
\prompt{In}{incolor}{38}{\boxspacing}
\begin{Verbatim}[commandchars=\\\{\}]
\PY{n}{wave} \PY{o}{=} \PY{n}{Wave}\PY{p}{(}\PY{n}{total}\PY{o}{.}\PY{n}{values}\PY{p}{)}
\PY{n}{wave}\PY{o}{.}\PY{n}{plot}\PY{p}{(}\PY{p}{)}
\end{Verbatim}
\end{tcolorbox}

    \begin{center}
    \adjustimage{max size={0.9\linewidth}{0.9\paperheight}}{output_70_0.png}
    \end{center}
    { \hspace*{\fill} \\}
    
    Here's the whole process in a function:

    \begin{tcolorbox}[breakable, size=fbox, boxrule=1pt, pad at break*=1mm,colback=cellbackground, colframe=cellborder]
\prompt{In}{incolor}{39}{\boxspacing}
\begin{Verbatim}[commandchars=\\\{\}]
\PY{k}{def} \PY{n+nf}{voss}\PY{p}{(}\PY{n}{nrows}\PY{p}{,} \PY{n}{ncols}\PY{o}{=}\PY{l+m+mi}{16}\PY{p}{)}\PY{p}{:}
    \PY{l+s+sd}{\PYZdq{}\PYZdq{}\PYZdq{}Generates pink noise using the Voss\PYZhy{}McCartney algorithm.}
\PY{l+s+sd}{    }
\PY{l+s+sd}{    nrows: number of values to generate}
\PY{l+s+sd}{    rcols: number of random sources to add}
\PY{l+s+sd}{    }
\PY{l+s+sd}{    returns: NumPy array}
\PY{l+s+sd}{    \PYZdq{}\PYZdq{}\PYZdq{}}
    \PY{n}{array} \PY{o}{=} \PY{n}{np}\PY{o}{.}\PY{n}{empty}\PY{p}{(}\PY{p}{(}\PY{n}{nrows}\PY{p}{,} \PY{n}{ncols}\PY{p}{)}\PY{p}{)}
    \PY{n}{array}\PY{o}{.}\PY{n}{fill}\PY{p}{(}\PY{n}{np}\PY{o}{.}\PY{n}{nan}\PY{p}{)}
    \PY{n}{array}\PY{p}{[}\PY{l+m+mi}{0}\PY{p}{,} \PY{p}{:}\PY{p}{]} \PY{o}{=} \PY{n}{np}\PY{o}{.}\PY{n}{random}\PY{o}{.}\PY{n}{random}\PY{p}{(}\PY{n}{ncols}\PY{p}{)}
    \PY{n}{array}\PY{p}{[}\PY{p}{:}\PY{p}{,} \PY{l+m+mi}{0}\PY{p}{]} \PY{o}{=} \PY{n}{np}\PY{o}{.}\PY{n}{random}\PY{o}{.}\PY{n}{random}\PY{p}{(}\PY{n}{nrows}\PY{p}{)}
    
    \PY{c+c1}{\PYZsh{} the total number of changes is nrows}
    \PY{n}{n} \PY{o}{=} \PY{n}{nrows}
    \PY{n}{cols} \PY{o}{=} \PY{n}{np}\PY{o}{.}\PY{n}{random}\PY{o}{.}\PY{n}{geometric}\PY{p}{(}\PY{l+m+mf}{0.5}\PY{p}{,} \PY{n}{n}\PY{p}{)}
    \PY{n}{cols}\PY{p}{[}\PY{n}{cols} \PY{o}{\PYZgt{}}\PY{o}{=} \PY{n}{ncols}\PY{p}{]} \PY{o}{=} \PY{l+m+mi}{0}
    \PY{n}{rows} \PY{o}{=} \PY{n}{np}\PY{o}{.}\PY{n}{random}\PY{o}{.}\PY{n}{randint}\PY{p}{(}\PY{n}{nrows}\PY{p}{,} \PY{n}{size}\PY{o}{=}\PY{n}{n}\PY{p}{)}
    \PY{n}{array}\PY{p}{[}\PY{n}{rows}\PY{p}{,} \PY{n}{cols}\PY{p}{]} \PY{o}{=} \PY{n}{np}\PY{o}{.}\PY{n}{random}\PY{o}{.}\PY{n}{random}\PY{p}{(}\PY{n}{n}\PY{p}{)}

    \PY{n}{df} \PY{o}{=} \PY{n}{pd}\PY{o}{.}\PY{n}{DataFrame}\PY{p}{(}\PY{n}{array}\PY{p}{)}
    \PY{n}{df}\PY{o}{.}\PY{n}{fillna}\PY{p}{(}\PY{n}{method}\PY{o}{=}\PY{l+s+s1}{\PYZsq{}}\PY{l+s+s1}{ffill}\PY{l+s+s1}{\PYZsq{}}\PY{p}{,} \PY{n}{axis}\PY{o}{=}\PY{l+m+mi}{0}\PY{p}{,} \PY{n}{inplace}\PY{o}{=}\PY{k+kc}{True}\PY{p}{)}
    \PY{n}{total} \PY{o}{=} \PY{n}{df}\PY{o}{.}\PY{n}{sum}\PY{p}{(}\PY{n}{axis}\PY{o}{=}\PY{l+m+mi}{1}\PY{p}{)}

    \PY{k}{return} \PY{n}{total}\PY{o}{.}\PY{n}{values}
\end{Verbatim}
\end{tcolorbox}

    To test it I'll generate 11025 values:

    \begin{tcolorbox}[breakable, size=fbox, boxrule=1pt, pad at break*=1mm,colback=cellbackground, colframe=cellborder]
\prompt{In}{incolor}{40}{\boxspacing}
\begin{Verbatim}[commandchars=\\\{\}]
\PY{n}{ys} \PY{o}{=} \PY{n}{voss}\PY{p}{(}\PY{l+m+mi}{11025}\PY{p}{)}
\PY{n}{ys}
\end{Verbatim}
\end{tcolorbox}

            \begin{tcolorbox}[breakable, size=fbox, boxrule=.5pt, pad at break*=1mm, opacityfill=0]
\prompt{Out}{outcolor}{40}{\boxspacing}
\begin{Verbatim}[commandchars=\\\{\}]
array([6.82730871, 8.2586287 , 7.97144041, {\ldots}, 9.07108066, 8.36198061,
       8.76904708])
\end{Verbatim}
\end{tcolorbox}
        
    And make them into a Wave:

    \begin{tcolorbox}[breakable, size=fbox, boxrule=1pt, pad at break*=1mm,colback=cellbackground, colframe=cellborder]
\prompt{In}{incolor}{41}{\boxspacing}
\begin{Verbatim}[commandchars=\\\{\}]
\PY{n}{wave} \PY{o}{=} \PY{n}{Wave}\PY{p}{(}\PY{n}{ys}\PY{p}{)}
\PY{n}{wave}\PY{o}{.}\PY{n}{unbias}\PY{p}{(}\PY{p}{)}
\PY{n}{wave}\PY{o}{.}\PY{n}{normalize}\PY{p}{(}\PY{p}{)}
\end{Verbatim}
\end{tcolorbox}

    Here's what it looks like:

    \begin{tcolorbox}[breakable, size=fbox, boxrule=1pt, pad at break*=1mm,colback=cellbackground, colframe=cellborder]
\prompt{In}{incolor}{42}{\boxspacing}
\begin{Verbatim}[commandchars=\\\{\}]
\PY{n}{wave}\PY{o}{.}\PY{n}{plot}\PY{p}{(}\PY{p}{)}
\end{Verbatim}
\end{tcolorbox}

    \begin{center}
    \adjustimage{max size={0.9\linewidth}{0.9\paperheight}}{output_78_0.png}
    \end{center}
    { \hspace*{\fill} \\}
    
    As expected, it is more random-walk-like than white noise, but more
random looking than red noise.

Here's what it sounds like:

    \begin{tcolorbox}[breakable, size=fbox, boxrule=1pt, pad at break*=1mm,colback=cellbackground, colframe=cellborder]
\prompt{In}{incolor}{43}{\boxspacing}
\begin{Verbatim}[commandchars=\\\{\}]
\PY{n}{wave}\PY{o}{.}\PY{n}{make\PYZus{}audio}\PY{p}{(}\PY{p}{)}
\end{Verbatim}
\end{tcolorbox}

            \begin{tcolorbox}[breakable, size=fbox, boxrule=.5pt, pad at break*=1mm, opacityfill=0]
\prompt{Out}{outcolor}{43}{\boxspacing}
\begin{Verbatim}[commandchars=\\\{\}]
<IPython.lib.display.Audio object>
\end{Verbatim}
\end{tcolorbox}
        
    And here's the power spectrum:

    \begin{tcolorbox}[breakable, size=fbox, boxrule=1pt, pad at break*=1mm,colback=cellbackground, colframe=cellborder]
\prompt{In}{incolor}{44}{\boxspacing}
\begin{Verbatim}[commandchars=\\\{\}]
\PY{n}{spectrum} \PY{o}{=} \PY{n}{wave}\PY{o}{.}\PY{n}{make\PYZus{}spectrum}\PY{p}{(}\PY{p}{)}
\PY{n}{spectrum}\PY{o}{.}\PY{n}{hs}\PY{p}{[}\PY{l+m+mi}{0}\PY{p}{]} \PY{o}{=} \PY{l+m+mi}{0}
\PY{n}{spectrum}\PY{o}{.}\PY{n}{plot\PYZus{}power}\PY{p}{(}\PY{p}{)}
\PY{n}{decorate}\PY{p}{(}\PY{n}{xlabel}\PY{o}{=}\PY{l+s+s1}{\PYZsq{}}\PY{l+s+s1}{Frequency (Hz)}\PY{l+s+s1}{\PYZsq{}}\PY{p}{,}
         \PY{n}{ylabel}\PY{o}{=}\PY{l+s+s1}{\PYZsq{}}\PY{l+s+s1}{Power}\PY{l+s+s1}{\PYZsq{}}\PY{p}{,}
         \PY{o}{*}\PY{o}{*}\PY{n}{loglog}\PY{p}{)}
\end{Verbatim}
\end{tcolorbox}

    \begin{center}
    \adjustimage{max size={0.9\linewidth}{0.9\paperheight}}{output_82_0.png}
    \end{center}
    { \hspace*{\fill} \\}
    
    The estimated slope is close to -1.

    \begin{tcolorbox}[breakable, size=fbox, boxrule=1pt, pad at break*=1mm,colback=cellbackground, colframe=cellborder]
\prompt{In}{incolor}{45}{\boxspacing}
\begin{Verbatim}[commandchars=\\\{\}]
\PY{n}{spectrum}\PY{o}{.}\PY{n}{estimate\PYZus{}slope}\PY{p}{(}\PY{p}{)}\PY{o}{.}\PY{n}{slope}
\end{Verbatim}
\end{tcolorbox}

            \begin{tcolorbox}[breakable, size=fbox, boxrule=.5pt, pad at break*=1mm, opacityfill=0]
\prompt{Out}{outcolor}{45}{\boxspacing}
\begin{Verbatim}[commandchars=\\\{\}]
-1.0337986006088606
\end{Verbatim}
\end{tcolorbox}
        
    We can get a better sense of the average power spectrum by generating a
longer sample:

    \begin{tcolorbox}[breakable, size=fbox, boxrule=1pt, pad at break*=1mm,colback=cellbackground, colframe=cellborder]
\prompt{In}{incolor}{46}{\boxspacing}
\begin{Verbatim}[commandchars=\\\{\}]
\PY{n}{seg\PYZus{}length} \PY{o}{=} \PY{l+m+mi}{64} \PY{o}{*} \PY{l+m+mi}{1024}
\PY{n}{iters} \PY{o}{=} \PY{l+m+mi}{100}
\PY{n}{wave} \PY{o}{=} \PY{n}{Wave}\PY{p}{(}\PY{n}{voss}\PY{p}{(}\PY{n}{seg\PYZus{}length} \PY{o}{*} \PY{n}{iters}\PY{p}{)}\PY{p}{)}
\PY{n+nb}{len}\PY{p}{(}\PY{n}{wave}\PY{p}{)}
\end{Verbatim}
\end{tcolorbox}

            \begin{tcolorbox}[breakable, size=fbox, boxrule=.5pt, pad at break*=1mm, opacityfill=0]
\prompt{Out}{outcolor}{46}{\boxspacing}
\begin{Verbatim}[commandchars=\\\{\}]
6553600
\end{Verbatim}
\end{tcolorbox}
        
    And using Barlett's method to compute the average.

    \begin{tcolorbox}[breakable, size=fbox, boxrule=1pt, pad at break*=1mm,colback=cellbackground, colframe=cellborder]
\prompt{In}{incolor}{47}{\boxspacing}
\begin{Verbatim}[commandchars=\\\{\}]
\PY{n}{spectrum} \PY{o}{=} \PY{n}{bartlett\PYZus{}method}\PY{p}{(}\PY{n}{wave}\PY{p}{,} \PY{n}{seg\PYZus{}length}\PY{o}{=}\PY{n}{seg\PYZus{}length}\PY{p}{,} \PY{n}{win\PYZus{}flag}\PY{o}{=}\PY{k+kc}{False}\PY{p}{)}
\PY{n}{spectrum}\PY{o}{.}\PY{n}{hs}\PY{p}{[}\PY{l+m+mi}{0}\PY{p}{]} \PY{o}{=} \PY{l+m+mi}{0}
\PY{n+nb}{len}\PY{p}{(}\PY{n}{spectrum}\PY{p}{)}
\end{Verbatim}
\end{tcolorbox}

            \begin{tcolorbox}[breakable, size=fbox, boxrule=.5pt, pad at break*=1mm, opacityfill=0]
\prompt{Out}{outcolor}{47}{\boxspacing}
\begin{Verbatim}[commandchars=\\\{\}]
32769
\end{Verbatim}
\end{tcolorbox}
        
    It's pretty close to a straight line, with some curvature at the highest
frequencies.

    \begin{tcolorbox}[breakable, size=fbox, boxrule=1pt, pad at break*=1mm,colback=cellbackground, colframe=cellborder]
\prompt{In}{incolor}{48}{\boxspacing}
\begin{Verbatim}[commandchars=\\\{\}]
\PY{n}{spectrum}\PY{o}{.}\PY{n}{plot\PYZus{}power}\PY{p}{(}\PY{p}{)}
\PY{n}{decorate}\PY{p}{(}\PY{n}{xlabel}\PY{o}{=}\PY{l+s+s1}{\PYZsq{}}\PY{l+s+s1}{Frequency (Hz)}\PY{l+s+s1}{\PYZsq{}}\PY{p}{,}
         \PY{n}{ylabel}\PY{o}{=}\PY{l+s+s1}{\PYZsq{}}\PY{l+s+s1}{Power}\PY{l+s+s1}{\PYZsq{}}\PY{p}{,}
         \PY{o}{*}\PY{o}{*}\PY{n}{loglog}\PY{p}{)}
\end{Verbatim}
\end{tcolorbox}

    \begin{center}
    \adjustimage{max size={0.9\linewidth}{0.9\paperheight}}{output_90_0.png}
    \end{center}
    { \hspace*{\fill} \\}
    
    And the slope is close to -1.

    \begin{tcolorbox}[breakable, size=fbox, boxrule=1pt, pad at break*=1mm,colback=cellbackground, colframe=cellborder]
\prompt{In}{incolor}{49}{\boxspacing}
\begin{Verbatim}[commandchars=\\\{\}]
\PY{n}{spectrum}\PY{o}{.}\PY{n}{estimate\PYZus{}slope}\PY{p}{(}\PY{p}{)}\PY{o}{.}\PY{n}{slope}
\end{Verbatim}
\end{tcolorbox}

            \begin{tcolorbox}[breakable, size=fbox, boxrule=.5pt, pad at break*=1mm, opacityfill=0]
\prompt{Out}{outcolor}{49}{\boxspacing}
\begin{Verbatim}[commandchars=\\\{\}]
-1.001210590869293
\end{Verbatim}
\end{tcolorbox}
        
    \begin{tcolorbox}[breakable, size=fbox, boxrule=1pt, pad at break*=1mm,colback=cellbackground, colframe=cellborder]
\prompt{In}{incolor}{ }{\boxspacing}
\begin{Verbatim}[commandchars=\\\{\}]

\end{Verbatim}
\end{tcolorbox}


    % Add a bibliography block to the postdoc
    
    
    
\end{document}
